\documentclass[a4paper,oneside]{book}
\input{../Mod_base/base}
\geometry{top=2cm,bottom=2cm,left=2cm,right=2cm}
\input{../Mod_base/grafica}


\input{../Mod_base/matematica}
\input{../Mod_base/tabelle}
\DeclareCaptionFormat{grafico}{\textbf{Grafico \thefigure}#2#3}
\DeclareCaptionFormat{esempio}{\textbf{Esempio \thefigure}#2#3}
\usepackage{imakeidx}
\makeindex[options=-s ../Mod_base/oldclaudio.sti]
\input{../Mod_base/pagina}
\input{../Mod_base/indice}
%\titlecontents{figure}
%[0em]{}
%{\thecontentslabel\hspace*{1.5em}}
%{}{\titlerule*[3pt]{}\contentspage}
%\titlecontents{table}
%[0em]{}
%{\thecontentslabel\hspace*{1.5em}}
%{}{\ \titlerule*[3pt]{}\contentspage}
%\titlecontents{figure}%
%[0em]% 
%{\addvspace{2em}}%
%{\bfseries\@chapapp\ \thecontentslabel\quad}% 
%{\hspace{-0em}}% 
%{\hfill\contentspage}% 
%[\addvspace{0pt}]% 

\input{../Mod_base/date}
\input{../Mod_base/loghi}
\input{../Mod_base/unita_misura}
\input{../Mod_base/utili}
\input{../Mod_base/stand_class}
%per le semplificazioni

\newcommand{\HRule}{\rule{\linewidth}{0.5mm}}
\usepackage{placeins} 
\makeatletter
\renewcommand\frontmatter{%
	\cleardoublepage
	\@mainmatterfalse
	\pagenumbering{arabic}}
\renewcommand\mainmatter{%
	\cleardoublepage
	\@mainmattertrue}
\makeatother
\listfiles
%%%%%%%%%%%%%%%%%%%%%%%%%%%%%%%%
%%%lunghezza arrotondamenti%%%%%
\newcommand{\lungarrotandamento}{4}
%%%%%%%%%%%%%%%%%%%%%%%%%%%%%%%%%%
\usepackage[grumpy,mark,markifdirty,raisemark=0.95\paperheight]{gitinfo2}
\usepackage[toc,page]{appendix}

\renewcommand{\appendixtocname}{Appendice}

\renewcommand{\appendixpagename}{Appendice}
\usepackage{tkz-berge}
\usepackage[italian]{varioref}
\usepackage{hyperxmp}
\usepackage[pdfpagelabels]{hyperref}
\usepackage[italian]{cleveref}
\input{../Mod_base/tcolorboxgest}
\title{Esercizi svolti terzo moda}
\author{Claudio Duchi}
\date{\datetime}
\hypersetup{%
	pdfencoding=auto,
	urlcolor={blue},
	pdftitle={Esercizi svolti},
	pdfsubject={terzo moda},
	pdfstartview={FitH},
	pdfpagemode={UseOutlines},
	pdflicenseurl={http://creativecommons.org/licenses/by-nc-nd/3.0/},
	pdflang={it},
	pdfmetalang={it},
	pdfkeywords={goniometria, trigoniometria, geometria analitica},
	pdfcopyright={Copyright (C) 2018, Claudio Duchi},
	pdfcontacturl={http://breviariomatematico.altervista.org},
	pdfcontactpostcode={},
	pdfcontactphone={},
	pdfcontactemail={claduc},
	pdfcontactcountry={Italy},
	pdfcontactcity={Perugia},
	pdfcontactaddress={},
	pdfcaptionwriter={Claudio Duchi},
	pdfauthortitle={},%
	pdfauthor={Claudio Duchi},
	linkcolor={blue},
	colorlinks=true,
	citecolor={red},
	breaklinks,
	bookmarksopen,
	verbose,
	baseurl={http://breviariomatematico.altervista.org}
}
\includeonly{%
terzo/DistTraDuePunti,
terzo/puntomedio,
terzo/retta, 
terzo/parabola, 
terzo/eserciziriepilogo,}
\begin{document}
\frontmatter
		\begin{titlepage}
	\begin{center}
		\input{../Mod_base/Lgrande}\\[1cm]
		\textsc{\LARGE Claudio Duchi}\\[1.5cm]
		\HRule \\[0.4cm]
		{ \huge \bfseries ESERCIZI SVOLTI DI MATEMATICA}\\[0.4cm]
		{\LARGE \textsc{TERZO MODA}}
		\HRule \\[1.5cm]
		\vfill
%	\end{center}
%	\begin{center}
		\begin{tikzpicture}
		\renewcommand*{\VertexBallColor}{green!50!black}
		\GraphInit[vstyle=Art]
		\grComplete[RA=5]{18}
		\end{tikzpicture}
	\end{center}
	{\centering
	Release:\gitReln\ (\gitAbbrevHash)\ Autore:\gitAuthorName\ 
	\gitCommitterDate \\
}
\end{titlepage}
\setcounter{page}{2}
\input{../Mod_base/copyright}
\tableofcontents 
%\addcontentsline{toc}{chapter}{\listtablename}
%\listoftables
\addcontentsline{toc}{chapter}{\listfigurename}
\listoffigures
\renewcommand\lstlistlistingname{Esempi e contro esempi}
\addcontentsline{toc}{chapter}{\lstlistlistingname}
\addcontentsline{toc}{section}{Esempi}
\lstlistoflistings%{}
{
%	https://tex.stackexchange.com/questions/318486/number-freestyle-causes-an-overlay-in-the-list-of-tcolorboxes/318512#318512
	\makeatletter
	\renewcommand{\l@tcolorbox}{\@dottedtocline{1}{0pt}{3em}}
\tcblistof[\section*]{thm}{Esempi}
\addcontentsline{toc}{section}{Contro esempi}
\tcblistof[\section*]{cthm}{Contro esempi}
}

\mainmatter%
\chapter{Distanza tra due punti}
\label{cha:DistanzaTraduePunti}
\begin{esempiot}{Distanza con stessa ordinata}{disuno}
Dati i punti $A\coord{1}{2}$\,$B\coord{3}{2}$ calcolare la distanza tra $A$ e $B$
\end{esempiot}
I due punti hanno la stessa ordinata quindi:
\begin{equation*}
d(AB)=\abs{x_1-x_2}=\abs{1-3}=\abs{-2}=2
\end{equation*}
\begin{esempiot}{Distanza con stessa ascissa}{distre}
	Dati i punti $A\coord{2}{4}$\,$B\coord{2}{7}$ calcolare la distanza tra $A$ e $B$
\end{esempiot}
I due punti hanno la stessa ascissa quindi: 
\begin{equation*}
	d(AB)=\abs{y_1-y_2}=\abs{4-7}=\abs{-3}=3
\end{equation*}
\begin{esempiot}{Distanza con stessa ordinata}{disdue}
	Dati i punti $A\coord{3}{-5}$\,$B\coord{-6}{-5}$ calcolare la distanza tra $A$ e $B$
\end{esempiot}
I due punti hanno la stessa ordinata quindi: 
\begin{equation*}
d(AB)=\abs{x_1-x_2}=\abs{3-(-6)}=\abs{3+6}=\abs{9}=9
\end{equation*}

\begin{esempiot}{Distanza con stessa ascissa}{disquattro}
	Dati i punti $A\coord{6}{-5}$\,$B\coord{6}{-2}$ calcolare la distanza tra $A$ e $B$
\end{esempiot}
I due punti hanno la stessa ascissa quindi: 
\begin{equation*}
d(AB)=\abs{y_1-y_2}=\abs{-5-(-2)}=\abs{-5+2}=\abs{-3}=3
\end{equation*}
\begin{esempiot}{Distanza caso generale}{dis5}
	Dati i punti $A\coord{3}{5}$\,$B\coord{4}{2}$ calcolare la distanza tra $A$ e $B$
\end{esempiot}
I due punti hanno la stessa ascissa quindi: 
\begin{align*}
	d(AB)=&\sqrt{(x_1-x_2)^2+(y_1-y_2)^2}\\
	=&\sqrt{(3-4)^2+(5-2)^2}\\
=&\sqrt{(-1)^2+(3)^2}\\
=&\sqrt{1+9}\\
=&\sqrt{10}
\end{align*}

\begin{esempiot}{Distanza caso generale}{discinque}
	Dati i punti $A\coord{2}{-4}$\,$B\coord{-5}{6}$ calcolare la distanza tra $A$ e $B$
\end{esempiot}
I due punti hanno la stessa ascissa quindi: 
\begin{align*}
d(AB)=&\sqrt{(x_1-x_2)^2+(y_1-y_2)^2}\\
=&\sqrt{(2-(-5))^2+(-4-6)^2}\\
=&\sqrt{(2+5)^2+(-10)^2}\\
=&\sqrt{49+100}\\
=&\sqrt{149}
\end{align*}


\chapter{Punto Medio}
\begin{esempiot}{Calcolo del punto medio}{pmedio1}
	Dati i punti $A\coord{2}{5}$\,$B\coord{4}{3}$ calcolare le coordinate del punto medio $M$
\end{esempiot}
\begin{equation*}
\begin{cases}
	x_m=\dfrac{x_1+x_2}{2}=\dfrac{2+4}{2}=\dfrac{6}{2}=3\\[0.8cm]
	y_m=\dfrac{y_1+y_2}{2}=\dfrac{5+3}{2}\dfrac{8}{2}=4
\end{cases}
\end{equation*}
\chapter{Retta}
\label{cha:retta}
\section{Disegnare una retta}
\begin{esempiot}{Retta nota}{retta1}
	Disegnare il grafico della retta $y=x+1$
\end{esempiot}
\begin{enumerate}
	\item Costruiamo la tabella a doppia entrata 
		\begin{tabular}{c|c}
		x & y\\
		\hline 
		&  \\ 
		&  \\ 
	\end{tabular}
	\item Si attribuisce  un valore alla $x$ e si completa la tabella
	$y=1+1=2$
	\begin{tabular}{c|c}
		x & y\\
		\hline 
		1	& 2 \\ 
		&  \\ 
	\end{tabular}
	\item Si da un altro valore alla $x$ si ottiene 
	$y=2+1=3$
	\begin{tabular}{c|c}
		x & y\\
		\hline 
		1	& 2 \\ 
		2& 3 \\ 
	\end{tabular}
	\item Ottengo le coppie $A\coord{1}{2}$\,$B\coord{2}{3}$ che unite formano il grafico~\vref{fig:disegnoretta1}
\end{enumerate}
\begin{center}
\includestandalone[width=.6\textwidth]{terzo/grafici/retta_dis_1}
\captionof{figure}{Due punti una retta}\label{fig:disegnoretta1}
\end{center}
\begin{esempiot}{Retta nota}{retta2}
	Disegnare il grafico della retta $y=-2x+1$
\end{esempiot}
\begin{enumerate}
	\item Costruiamo la tabella a doppia entrata 
	\begin{tabular}{c|c}
		x & y\\
		\hline 
		&  \\ 
		&  \\ 
	\end{tabular}
	\item Si attribuisce  un valore alla $x$ e si completa la tabella
	$y=-2+1=-1$
	\begin{tabular}{c|c}
		x & y\\
		\hline 
		1	& -1 \\ 
		&  \\ 
	\end{tabular}
	\item Si da un altro valore alla $x$ si ottiene 
	$y=-2(-1)+1=2+1=3$
	\begin{tabular}{c|c}
		x & y\\
		\hline 
		1	& 2 \\ 
		-1& 3 \\ 
	\end{tabular}
	\item Ottengo le coppie $A\coord{1}{2}$\,$B\coord{2}{3}$ che unite formano il grafico~\vref{fig:disegnoretta2}
\end{enumerate}
\begin{center}
	\includestandalone[width=.5\textwidth]{terzo/grafici/retta_dis_2}
	\captionof{figure}{Due punti una retta}\label{fig:disegnoretta2}
\end{center}
\begin{esempiot}{Retta nota parallela all'asse $x$}{retta3}
	Disegnare il grafico della retta $y=2$
\end{esempiot}
\begin{enumerate}
	\item Costruiamo la tabella a doppia entrata 
	\begin{tabular}{c|c}
		x & y\\
		\hline 
		1&2\\ 
		2&2\\ 
	\end{tabular}
	\item Ottengo le coppie $A\coord{1}{2}$\,$B\coord{2}{2}$ che unite formano il grafico~\vref{fig:disegnoretta3}
\end{enumerate}
\begin{center}
	\includestandalone[width=.6\textwidth]{terzo/grafici/retta_dis_3}
	\captionof{figure}{Due punti una retta}\label{fig:disegnoretta3}
\end{center}
\begin{esempiot}{Retta nota parallela asse $y$}{retta4}
	Disegnare il grafico della retta $x=2$
\end{esempiot}
\begin{enumerate}
	\item Costruiamo la tabella a doppia entrata 
	\begin{tabular}{c|c}
		x & y\\
		\hline 
		2&1\\ 
		2&2\\ 
	\end{tabular}
	\item Ottengo le coppie $A\coord{2}{1}$\,$B\coord{2}{2}$ che unite formano il grafico~\vref{fig:disegnoretta4}
\end{enumerate}
\begin{center}
	\includestandalone[width=.6\textwidth]{terzo/grafici/retta_dis_4}
	\captionof{figure}{Due punti una retta}\label{fig:disegnoretta4}
\end{center}
\section{Passaggio per un punto}
\begin{esempiot}{Verificare se una retta passa per un punto}{punto1}
Data la retta $y=3x+5$ Verificare se la retta passa per i punti $A\coord{1}{8}$\,$B\coord{-1}{3}$
\end{esempiot}
Verifico se la retta passa per il punto $A\coord{1}{8}$ 
\begin{enumerate}
	\item Sostituisco $A\coord{1}{8}$ nella retta $y=3x+5$ 
	\item Ottengo  \begin{tabular}{rll}
		$8=$&$+3\cdot 1$ &$+5$  \\ 
		$8=$&$3+5$  \\ 
		$8=$&$8$  \\ 
	\end{tabular}
\end{enumerate}	
	La retta passa per $A$
	
Verifico se la retta passa per il punto $B\coord{-1}{3}$
\begin{enumerate}
	\item Sostituisco $B\coord{-1}{3}$ nella retta $y=3x+5$ 
	\item Ottengo  \begin{tabular}{rll}
		$3=$&$+3\cdot (-1)$ &$+5$  \\ 
		$3=$&$-3+5$  \\ 
		$8=$&$2$  \\ 
	\end{tabular}
\end{enumerate}	
La retta non passa per $B$ 
\section{Retta per due punti}
\begin{esempiot}{Dati due punti trovare l'equazione della retta}{retta5}
Data la coppia di punti	$A\coord{1}{3}$\,$B\coord{-1}{1}$ trovare l'equazione della retta che passa per questi punti.
\end{esempiot}
\begin{enumerate}
	\item Consideriamo la retta generica $y=mx+q$
	\item Passaggio per $A\coord{1}{3}$ otteniamo $3=1\cdot m+q$
	\item Passaggio per $B\coord{-1}{1}$ otteniamo $1=-1\cdot m+q$
	\item Allineo i due risultati e sottraggo
	\begin{tabular}{rll}
	$3=$&$+1\cdot m$ &$+q$  \\ 
	$1=$&$-1\cdot m$ &$+q$  \\ 
	\hline  $2=$&$+2\cdot m$& $0$ \\ 
	\end{tabular} 
	\item Semplifico e otteniamo $m=1$
	\item Prendo una delle precedenti relazioni e sostituisco il valore di $m=1$ trovato
	\item \begin{tabular}{rll}
			$3=$&$+1\cdot 1$ &$+q$  \\ 
			$3=$&$+1$ &$+q$  \\ 
			$3-1=$& $+q$  \\ 
			$q=$&$+2$ 
		\end{tabular} 
	\item Quindi $m=1$ $q=2$ che sostituiti in $y=mx+q$ otteniamo l'equazione cercata $y=x+2$
\end{enumerate}
\begin{esempiot}{Dati due punti trovare l'equazione della retta}{retta6}
	Data la coppia di punti	$A\coord{-2}{3}$\,$B\coord{3}{-1}$ trovare l'equazione della retta che passa per questi punti.
\end{esempiot}
\begin{enumerate}
	\item Consideriamo la retta generica $y=mx+q$
	\item Passaggio per $A\coord{-2}{3}$ otteniamo $3=-2\cdot m+q$
	\item Passaggio per $B\coord{3}{-1}$ otteniamo $-1=3\cdot m+q$
	\item Allineo i due risultati e sottraggo
	\begin{tabular}{rll}
		$+3=$&$-2\cdot m$ &$+q$  \\ 
		$-1=$&$+3\cdot m$ &$+q$  \\ 
		\hline $+4=$&$-5\cdot m$& $0$ \\ 
	\end{tabular} 
	\item Semplifico e otteniamo $m=-\dfrac{4}{5}$
	\item Prendo una delle precedenti relazioni e sostituisco il valore di $m=-\dfrac{4}{5}$ trovato
	\item \begin{tabular}{rll}
		$-1=$&$+3\cdot(-\dfrac{4}{5}) $ &$+q$  \\ 
		$-1=$&$-\dfrac{12}{5}$ &$+q$  \\ 
		$-1+\dfrac{12}{5}=$& $+q$  \\ 
		$q=$&$\dfrac{7}{5}$ 
	\end{tabular} 
	\item Quindi $m=-\dfrac{4}{5}$ $q=\dfrac{7}{5}$ che sostituiti in $y=mx+q$ otteniamo l'equazione cercata $y=-\dfrac{4}{5}x+\dfrac{7}{5}$
\end{enumerate}
\begin{cesempiot}{Dati due punti trovare l'equazione della retta}{retta6a}
	Data la coppia di punti	$A\coord{3}{5}$\,$B\coord{3}{2}$ trovare l'equazione della retta che passa per questi punti.
\end{cesempiot}
\begin{enumerate}
	\item Consideriamo la retta generica $y=mx+q$
	\item Passaggio per $A\coord{3}{5}$ otteniamo $5=3\cdot m+q$
	\item Passaggio per $B\coord{3}{2}$ otteniamo $2=3\cdot m+q$
	\item Allineo i due risultati e sottraggo
	\begin{tabular}{rll}
		$5=$&$3\cdot m$ &$+q$  \\ 
		$2=$&$3\cdot m$ &$+q$  \\ 
		\midrule $3=$&$0$& $0$ \\ 
	\end{tabular} 
	\item Impossibile, il metodo non funziona.
	\item I punti $A$ e $B$ hanno la stessa ascissa.
	\item In questo caso l'equazione è $x=3$
\end{enumerate}
\section{Retta per un punto parallela a retta data}
\begin{esempiot}{Dato un punto e una retta parallela}{retta7}
	Data la retta $y=3x+4$ Trovare la retta parallela alla retta data che  passa per il punto	$A\coord{2}{3}$
\end{esempiot}
\begin{enumerate}
	\item Consideriamo la retta generica $y=mx+q$
	\item Le due rette sono parallele quindi $m=3$ otteniamo $y=3x+q$
	\item Passaggio per $A\coord{2}{3}$ otteniamo 
	\begin{tabular}{rll}
	$3=$&$3\cdot 2$&$+q$\\
	$3-6=$&$q$\\
	$q=$&$-3$\\
	\end{tabular}
\end{enumerate}

La retta cercata è $y=3x-3$ procedendo come con l'\cref{exa:retta1}
otteniamo
\begin{center}
	\includestandalone[width=.6\textwidth]{terzo/grafici/retta_dis_7}
	\captionof{figure}{Retta parallela a retta data}\label{fig:disegnoretta7}
\end{center}
\begin{esempiot}{Dato un punto e una retta parallela}{retta8}
	Data la retta $y=-3x+5$ Trovare la retta parallela alla retta data che  passa per il punto	$A\coord{-2}{4}$
\end{esempiot}
\begin{enumerate}
	\item Consideriamo la retta generica $y=mx+q$
	\item Le due rette sono parallele quindi $m=-3$ otteniamo $y=-3x+q$
	\item Passaggio per $A\coord{-2}{4}$ otteniamo 
	\begin{tabular}{rll}
		$4=$&$-3\cdot (-2)$&$+q$\\
		$4-6=$&$q$\\
		$q=$&$-2$\\
	\end{tabular}
\end{enumerate}

La retta cercata è $y=-3x-2$ procedendo come con l'\cref{exa:retta1}
otteniamo
\begin{center}
	\includestandalone[width=.6\textwidth]{terzo/grafici/retta_dis_8}
	\captionof{figure}{Retta parallela a retta data}\label{fig:disegnoretta8}
\end{center}
\begin{esempiot}{Dato un punto e una retta parallela}{retta8a}
	Trovare la retta parallela all'asse $x$ che passa per $A\coord{1}{2}$
\end{esempiot}
La retta cercata è $y=2$ procedendo come con l'\cref{exa:retta3}
otteniamo
\begin{center}
	\includestandalone[width=.6\textwidth]{terzo/grafici/retta_dis_8a}
	\captionof{figure}{Retta parallela all'asse $x$}\label{fig:disegnoretta8a}
\end{center}
\section{Retta per un punto perpendicolare a retta data}
\begin{esempiot}{Dato un punto e una retta perpendicolare}{retta9}
	Data la retta $y=2x+2$ Trovare la retta perpendicolare alla retta data che  passa per il punto	$A\coord{2}{4}$
\end{esempiot}
\begin{enumerate}
	\item Consideriamo la retta generica $y=m_2x+q$
	\item Le due rette sono perpendicolari quindi $2\cdot m_2=-1$ otteniamo $m_2=-\dfrac{1}{2}$ quindi $y=-\dfrac{1}{2}x+q$
	\item Passaggio per $A\coord{2}{4}$ otteniamo 
	\begin{tabular}{rll}
		$4=$&$-\dfrac{1}{2}\cdot 2$&$+q$\\
		$4+1=$&$q$\\
		$q=$&$5$\\
	\end{tabular}
\end{enumerate}

La retta cercata è $y=-\dfrac{1}{2}x+5$ procedendo come con l'\cref{exa:retta3}
otteniamo
\begin{center}
	\includestandalone[width=.6\textwidth]{terzo/grafici/retta_dis_9}
	\captionof{figure}{Retta perpendicolare a retta data}\label{fig:disegnoretta9}
\end{center}
\begin{esempiot}{Dato un punto e una retta perpendicolare}{retta10}
	Data la retta $y=-x+3$ Trovare la retta perpendicolare alla retta data che  passa per il punto	$A\coord{-1}{-1}$
\end{esempiot}
\begin{enumerate}
	\item Consideriamo la retta generica $y=m_2x+q$
	\item Le due rette sono perpendicolari quindi $-1\cdot m_2=-1$ otteniamo $m_2=1$ quindi $y=x+q$
	\item Passaggio per $A\coord{-1}{-1}$ otteniamo 
	\begin{tabular}{rll}
		$-1=$&$-1$&$+q$\\
		$-1+1=$&$q$\\
		$q=$&$0$\\
	\end{tabular}
\end{enumerate}

La retta cercata è $y=x$ procedendo come con l'\cref{exa:retta1}
otteniamo
\begin{center}
	\includestandalone[width=.6\textwidth]{terzo/grafici/retta_dis_10}
	\captionof{figure}{Retta perpendicolare a retta data}\label{fig:disegnoretta10}
\end{center}
\section{Fascio di rette}
\begin{esempiot}{Trovare l'equazione del fascio di centro un punto dato}{retta11}
	Dato il punto	$A\coord{3}{4}$ trovare l'equazione del fascio di centro $A$
\end{esempiot}
Partendo da $y-y_1=m(x-x_1)$ sostituisco le coordinate del centro  	$A\coord{3}{4}$
\begin{align*}
	y-4=&m(x-3)\\
	y=&m(x-3)+4
\end{align*}
L'equazione cercata è $y=m(x-3)+4$
\begin{esempiot}{Trovare l'equazione del fascio di centro un punto dato}{retta12}
	Dato il punto	$A\coord{3}{4}$ trovare l'equazione del fascio di centro $A$
\end{esempiot}
Partendo da $y-y_1=m(x-x_1)$ sostituisco le coordinate del centro  	$A\coord{-1}{-2}$
\begin{align*}
	y+2=&m(x+1)\\
	y=&m(x+1)-2
\end{align*}
L'equazione cercata è $y=m(x+1)$
\begin{esempiot}{Dati due punti trovare il coefficiente angolare della retta che passa per questi punti}{retta13}
	Dati i punti $A\coord{2}{5}$ e $B\coord{3}{7}$ trovare il coefficiente angolare della retta che passa per  $A$ E $B$
\end{esempiot}
\begin{align*}
	m=&\dfrac{y_2-y_1}{x_2-x_1}\\
	\intertext{Sostituisco le coordinate dei punti $A\coord{2}{5}$ e $B\coord{3}{7}$}
	m=&\dfrac{7-5}{3-2}\\
	m=&\dfrac{2}{1}\\
	m=&2\\
\end{align*}
Il coefficiente angolare è $m=2$
\begin{cesempiot}{Dati due punti trovare il coefficiente angolare della retta che passa per questi punti}{retta13a}
	Dati i punti $A\coord{2}{5}$ e $B\coord{2}{7}$ trovare il coefficiente angolare della retta che passa per  $A$ E $B$
\end{cesempiot}
\begin{align*}
	m=&\dfrac{y_2-y_1}{x_2-x_1}\\
	\intertext{Sostituisco  $A\coord{2}{5}$ e $B\coord{2}{7}$}
	m=&\dfrac{7-5}{2-2}\\
	m=&\dfrac{2}{0}\\
\intertext{Impossibile}
\end{align*}
Non esiste il coefficiente angolare.
\subsection{Retta per un punto parallela a retta data}
\begin{esempiot}{Dato un punto e una retta parallela}{retta14}
	Data la retta $y=3x+4$ Trovare la retta parallela alla retta data che  passa per il punto	$A\coord{2}{3}$
\end{esempiot}
Consideriamo l'equazione generica del fascio $y-y_1=m(x-x_1)$, se le due rette sono parallele $m=3$ quindi
\begin{align*}
	y-y_1=&3(x-x_1)\\
	\intertext{Coordinate del centro $A\coord{2}{3}$}
	y-3=&3(x-2)\\
	y=&3x-6+3\\
	y=&3x-3\\
\end{align*}
Otteniamo 	$y=3x-3$ lo stesso risultato dell'\cref{exa:retta7}
\subsection{Retta per un punto perpendicolare a retta data}
\begin{esempiot}{Dato un punto e una retta perpendicolare}{retta15}
	Data la retta $y=2x+2$ Trovare la retta perpendicolare alla retta data che  passa per il punto	$A\coord{2}{4}$
\end{esempiot}
Se le due rette sono perpendicolari allora $m_1\cdot m_2=-1$ quindi $2m_2=-1$ $m_2=-\dfrac{1}{2}$
Consideriamo l'equazione generica del fascio $y-y_1=m(x-x_1)$, 
\begin{align*}
	y-y_1=&-\dfrac{1}{2}(x-x_1)\\
	\intertext{Sostituisco le coordinate del centro $A\coord{2}{4}$}
	y-4=&-\dfrac{1}{2}(x-2)\\
	y=&-\dfrac{1}{2}x+1+4\\
	y=&-\dfrac{1}{2}x+5\\
\end{align*}
Otteniamo 	$y=-\dfrac{1}{2}x+5$ lo stesso risultato dell'\cref{exa:retta9}
\subsection{Retta per due punti}
\begin{esempiot}{Fascio di rette e retta per due punti}{rettaf1}
Dati due punti $A\coord{-5}{2}$ $B\coord{3}{6}$ trovare la retta che passa per questi punti.
\end{esempiot}
\begin{enumerate}
	\item Consideriamo l'equazione del fascio $y-y_1=m(x-x_1)$
	\item Sostituisco le coordinate del centro $A\coord{-5}{2}$ otteniamo $y-2=m(x+5)$
	\item Trovo il coefficiente angolare come con l'\cref{exa:retta13}
	\item $m=\dfrac{1}{2}$
	\item \begin{align*}
		y-2=&\dfrac{1}{2}(x+3)\\
		y=&\dfrac{1}{2}x+\dfrac{3}{2}+2\\
		y=&\dfrac{1}{2}x+\dfrac{7}{2}
	\end{align*}
\end{enumerate}
\begin{esempiot}{Fascio di rette e retta per due punti}{rettaf2}
	Dati due punti $A\coord{3}{6}$ $B\coord{5}{2}$ trovare la retta che passa per questi punti.
\end{esempiot}
\begin{enumerate}
	\item Consideriamo l'equazione del fascio $y-y_1=m(x-x_1)$
	\item Prendiamo  $A\coord{3}{6}$ come centro del fascio e otteniamo $y-6=m(x-3)$
	\item Passaggio per $B\coord{5}{2}$ otteniamo $2-6=m(5-3)$
	\item Semplifico $-4=m2$
	\item $m=-2$
	\item \begin{align*}
		y-6=&-2(x-3)\\
		y=&-2x+6+6\\
		y=&-2x+12
	\end{align*}
\end{enumerate}
\begin{cesempiot}{Fascio di rette e retta per due punti}{rettaf3a}
	Dati due punti $A\coord{2}{3}$ $B\coord{2}{5}$ trovare la retta che passa per questi punti.
\end{cesempiot}
\begin{enumerate}
	\item Consideriamo l'equazione del fascio $y-y_1=m(x-x_1)$
	\item Poniamo  $A\coord{2}{3}$ come centro del fascio e otteniamo $y-3=m(x-2)$
	\item Passaggio per $B\coord{2}{5}$ otteniamo $5-3=m(2-3)$
	\item Semplifico $2=0$
	\item Impossibile. In questo caso la soluzione è:
	\item $x=2$
\end{enumerate}
\begin{esempiot}{Retta per due punti}{rettaf3}
	Dati due punti $A\coord{3}{6}$ $B\coord{5}{2}$ trovare la retta che passa per questi punti.
\end{esempiot}
\begin{enumerate}
	\item Consideriamo l'equazione  $\dfrac{y-y_1}{y_2-y_1}=\dfrac{x-x_1}{x_2-x_1}$
	\item \begin{align*}
		\intertext{Passaggio per $A\coord{3}{6}$ $B\coord{5}{2}$ }
	\dfrac{y-4}{2-6}=&\dfrac{x-3}{5-3}\\
	\dfrac{y-4}{-4}=&\dfrac{x-3}{2}\\
		y-6=&-4\dfrac{x-3}{2}\\
		y=&-2(x-3)+6\\
		y=&-2x+12
	\end{align*}
\end{enumerate}
\section{Intersezioni fra rette}
\begin{esempiot}{Date due rette trovare il punto di intersezione}{retta16}
	Data le rette $y=-2x+4$ e $y=x+1$ trovare il loro eventuale punto di intersezione
\end{esempiot}
Imposto il sistema 
\[\begin{cases} 
	y=-2x+4\\
	y=+x+1
\end{cases}\]
Risolvo il sistema formato dalle due rette in forma esplicita con il metodo del confronto.
\begin{align*}
	&\begin{cases} 
		y=-2x+4\\
		y=+x+1
	\end{cases}&&\begin{cases} 
	x+1=-2x+4\\
	y=+x+1
\end{cases}\\
&\begin{cases} 
	3x=3\\
	y=x+1
\end{cases}&&\begin{cases} 
x=1\\
y=x+1
\end{cases}\\
&\begin{cases} 
	x=1\\
	y=1+1
\end{cases}&&
\begin{cases} 
	x=1\\
	y=2
\end{cases}
\end{align*}
Le due rette si incontrano in $A\coord{1}{2}$
\begin{esempiot}{Date due rette trovare il punto di intersezione}{retta17}
	Data le rette $y=\dfrac{1}{2}x+4$ e $y=-x+1$ trovare la loro eventuale intersezione.
\end{esempiot}
Imposto il sistema 
\[\begin{cases} 
y=+\dfrac{1}{2}x+4\\
y=-x+1
\end{cases}\]
Risolvo il sistema formato dalle due rette in forma esplicita con il metodo del confronto.
\begin{align*}
	&\begin{cases} 
		y=\dfrac{1}{2}x+4\\
		y=-x+1
	\end{cases}&&\begin{cases} 
	-x+1=\dfrac{1}{2}x+4\\
	y=-x+1
\end{cases}\\
&\begin{cases} 
	-2x+2=+x+8\\
	y=-x+1
\end{cases}&&\begin{cases} 
-2x-x=+8-2\\
y=-x+1
\end{cases}\\
&\begin{cases} 
	-3x=+6\\
	y=-x+1
\end{cases}&&
\begin{cases} 
	x=-2\\
	y=-x+1
\end{cases}\\
&\begin{cases} 
	x=-2\\
	y=3
\end{cases}
\end{align*}
Le due rette si incontrano in $A\coord{-2}{3}$
\begin{cesempiot}{Date due rette trovare il punto di intersezione}{retta18}
	Data le rette $y=2x+5$ e $y=2x+6$ trovare il loro eventuale punto di intersezione
\end{cesempiot}
Le due rette hanno lo stesso coefficiente angolare\index{Coefficiente!angolare} quindi non si incontrano, verifichiamo con il sistema 
\[\begin{cases} 
y=2x+5\\
y=2x+6
\end{cases}\]
Risolvo con il metodo del confronto.
\begin{align*}
	&\begin{cases} 
		y=2x+5\\
		y=2x+6
	\end{cases}&&\begin{cases} 
	2x+5=2x+6\\
	y=2x+6
\end{cases}\\
&\begin{cases} 
	5=6\\
	y=2x+6
\end{cases}
\end{align*}
Le due rette non si intersecano.
\begin{esempiot}{Date due rette trovare il punto di intersezione}{retta19}
	Data le rette $x+3y-4=0$ e $2x+3y-5=0$ trovare il loro eventuale punto di intersezione
\end{esempiot}
Imposto il sistema 
\[\begin{cases} 
x+3y=4\\
2x+3y=5
\end{cases}\]
Risolvo il sistema formato dalle due rette  con il metodo della sostituzione.
\begin{align*}
	&\begin{cases} 
	x+3y=4\\
	2x+3y=5
	\end{cases}&&\begin{cases} 
x=-3y+4\\
2x+3y=5
\end{cases}\\
&\begin{cases} 
	x=4-3y\\
	2(4-3y)+3y=5
\end{cases}&&\begin{cases} 
	x=4-3y\\
	8-6y+3y=5
\end{cases}\\
&\begin{cases} 
	x=4-3y\\
	-3y=5-8
\end{cases}&&\begin{cases} 
x=4-3y\\
-3y=-3
\end{cases}\\
&\begin{cases} 
	x=4-3y\\
	y=1
\end{cases}&&\begin{cases} 
x=4-3\\
y=1
\end{cases}\\
&\begin{cases} 
	x=1\\
	y=1
\end{cases}\\
\end{align*} 
\chapter{Parabola}
\label{cha:parabola}
\section{Disegnare una parabola}
\begin{esempiot}{Disegnare una parabola nota}{parabola1}
	Disegnare il grafico della parabola $y=4x^2-18x+18$
	\end{esempiot}
	La parabola ha $a>0$ quindi ha la concavità rivolta verso l'alto. La parabola ha l'asse parallelo all'asse $y$. L'asse ha quindi equazione\[x=-\dfrac{b}{2a}=-\dfrac{-18}{2\cdot 4}=\dfrac{18}{8}=\dfrac{9}{4}\]
	Per trovare l'ordinata del vertice $V$ sostituisco l'ascissa dell'asse nell'equazione della parabola
	\begin{align*}
y=&4x^2-18x+18\\
=&4\left( \dfrac{9}{4}\right)^2-18\dfrac{9}{4}+18\\
=&4\dfrac{81}{16}-18\dfrac{9}{4}+18\\
=&\dfrac{81}{4}-\dfrac{162}{4}+18\\
=&\dfrac{81-162+72}{4}\\
=&-\dfrac{9}{4}
	\end{align*}
	Il vertice ha quindi coordinate $V\coord{\dfrac{9}{4}}{-\dfrac{9}{4}}$
	
Passaggio per punti
\begin{align*}
x&=1\\
y=&4x^2-18x+18\\
=&4(1)^2-18\cdot1 +18\\
=&4
\end{align*}
\begin{align*}
x&=2\\
y=&4x^2-18x+18\\
=&4(2)^2-18\cdot2 +18\\
=&-2
\end{align*}

Ricapitolando
		\begin{tabular}{c|c}
			x & y\\
			\hline 
			1& 4 \\ 
			2&-2  \\ 
		\end{tabular}
			
$A\coord{1}{4}$ $B\coord{2}{-2}$ Per simmetria rispetto all'asse ottengo altri due punti.
\begin{center}
	\includestandalone[width=.5\textwidth]{terzo/grafici/parabola1}
	\captionof{figure}{Grafico parabola}\label{fig:disegnoparabola1}
\end{center}
\begin{esempiot}{Disegnare una parabola nota}{parabola2}
	Disegnare il grafico della parabola $y=-2x^2+8x-6$
\end{esempiot}

La parabola $a<0$ quindi ha la concavità rivolta verso il basso. Ha l'asse parallelo all'asse delle $y$. L'asse ha quindi equazione \[x=-\dfrac{b}{2a}=-\dfrac{8}{-4}=+2\]
Per trovare l'ordinata del vertice $V$ sostituisco l'ascissa dell'asse nell'equazione della parabola\[y=-2\cdot4+8\cdot2-6=2\]
Il vertice ha quindi coordinate $V\coord{2}{2}$
\begin{align*}
	x&=0\\
	y=&-2x^2+8x-6\\
	=&-2(0)^2+8\cdot0 -6\\
	=&-6
\end{align*}
\begin{align*}
	x&=1\\
	y=&-2x^2+8x-6\\
	=&-2(1)^2+8\cdot1 -6\\
	=&0
\end{align*}
Ricapitolando
\begin{tabular}{c|c}
	x & y\\
	\hline 
	0& -6 \\ 
	1&0  \\ 
\end{tabular}

$A\coord{0}{-6}$ $B\coord{1}{0}$ Per simmetria rispetto all'asse ottengo altri due punti.
\begin{center}
	\includestandalone[width=.5\textwidth]{terzo/grafici/parabola2}
	\captionof{figure}{Grafico parabola}\label{fig:disegnoparabola2}
\end{center}
\section{Elementi della parabola}	
\begin{esempiot}{Elementi di una parabola}{parabola3}
	Trovare asse, fuoco, vertice e direttrice della parabola $y=\dfrac{1}{4}x^2-x-2$
\end{esempiot}
$a>0$ la parabola ha concavità rivolta verso l'alto. I coefficienti sono\[\begin{cases}
a=\dfrac{1}{4}\\
b=-1\\
c=-2
\end{cases} \]
L'asse della parabola è\[x=-\dfrac{-b}{2a}=\dfrac{-1}{2\dfrac{-1}{4}}=2 \]
Per trovare le coordinate del fuoco $F\coord{-\dfrac{b}{2a}}{\dfrac{1-\Delta}{4a}}$, del vertice $V\coord{-\dfrac{b}{2a}}{-\dfrac{\Delta}{4a}}$ e l'equazione della direttrice $y=-\dfrac{1+\Delta}{4a}$, bisogna conoscere il valore di $\Delta$ e di $4a$. Inizio con il calcolare \[\Delta=b^2-4a=1-4\cdot\dfrac{1}{4}\cdot(-2)=3\]
\[4a=4\cdot\dfrac{1}{4}=1\]
Quindi \[\begin{cases}
\dfrac{1-\Delta}{4a}=\dfrac{1-3}{1}=-2\\
\\
-\dfrac{\Delta}{4a}=-\dfrac{3}{1}=-3\\
\\
-\dfrac{1+\Delta}{4a}=-\dfrac{1+3}{1}=-4
\end{cases} \]
Ricapitolando \[F\coord{2}{-2} \]\[V\coord{2}{-3} \]\[y=-4 \]
\begin{center}
	\includestandalone[width=.5\textwidth]{terzo/grafici/parabola3}
	\captionof{figure}{Elementi della parabola}\label{fig:disegnoparabola3}
\end{center}
\begin{esempiot}{Elementi di una parabola}{parabola3a}
	Trovare asse, fuoco, vertice e direttrice della parabola $y=-x^2+4$
\end{esempiot}
$a<0$ la parabola ha concavità rivolta verso il basso. I coefficienti sono\[\begin{cases}
a=-1\\
b=0\\
c=4
\end{cases} \]
L'asse della parabola è\[x=-\dfrac{-b}{2a}=\dfrac{0}{-2}=0 \]
Per trovare le coordinate del fuoco $F\coord{-\dfrac{b}{2a}}{\dfrac{1-\Delta}{4a}}$, del vertice $V\coord{-\dfrac{b}{2a}}{-\dfrac{\Delta}{4a}}$ e l'equazione della direttrice $y=-\dfrac{1+\Delta}{4a}$, bisogna conoscere il valore di $\Delta$ e di $4a$. Inizio con il calcolare \[\Delta=b^2-4a=0-4\cdot(-1)\cdot4=16\]
\[4a=4\cdot(-1)=-4\]
Quindi \[\begin{cases}
\dfrac{1-\Delta}{4a}=\dfrac{1-16}{-4}=\dfrac{-15}{-4}=\dfrac{15}{4}\\
\\
-\dfrac{\Delta}{4a}=-\dfrac{16}{-4}=\dfrac{16}{4}=4\\
\\
-\dfrac{1+\Delta}{4a}=-\dfrac{1+16}{-4}=\dfrac{17}{4}
\end{cases} \]
Ricapitolando \[F\coord{0}{\dfrac{15}{4}} \]\[V\coord{0}{3} \]\[y=\dfrac{17}{4} \]
\begin{center}
	\includestandalone[width=.5\textwidth]{terzo/grafici/parabola3a}
	\captionof{figure}{Elementi della parabola}\label{fig:disegnoparabola3a}
\end{center}
\begin{esempiot}{Elementi di una parabola}{parabola3b}
	Trovare asse, fuoco, vertice e direttrice della parabola $y=-\dfrac{3}{2}x^2+\dfrac{9}{2}$
\end{esempiot}
$a<0$ la parabola ha concavità rivolta verso il basso. I coefficienti sono\[\begin{cases}
a=-\dfrac{3}{2}\\
b=\dfrac{9}{2}\\
c=0
\end{cases} \]
L'asse della parabola è\[x=-\dfrac{-b}{2a}=-\dfrac{9}{2}\cdot\dfrac{1}{2(-\dfrac{3}{2})}=\dfrac{9}{2}\cdot\dfrac{1}{3}=\dfrac{3}{2} \]
Per trovare le coordinate del fuoco $F\coord{-\dfrac{b}{2a}}{\dfrac{1-\Delta}{4a}}$, del vertice $V\coord{-\dfrac{b}{2a}}{-\dfrac{\Delta}{4a}}$ e l'equazione della direttrice $y=-\dfrac{1+\Delta}{4a}$, bisogna conoscere il valore di $\Delta$ e di $4a$. Inizio con il calcolare \[\Delta=b^2-4a=\dfrac{81}{4}-4\cdot(-\dfrac{3}{2})\cdot0=\dfrac{81}{4}\]
\[4a=4\cdot\left(-\dfrac{3}{2}\right)=-6\]
Quindi \[\begin{cases}
\dfrac{1-\Delta}{4a}=\left(1-\dfrac{81}{4}\right)\left(-\dfrac{1}{6}\right)=\dfrac{4-81}{4}\left(-\dfrac{1}{6}\right)=\dfrac{77}{24}\\
\\
-\dfrac{\Delta}{4a}=-\dfrac{81}{4}\left(-\dfrac{1}{6}\right)=\dfrac{81}{24}=\dfrac{27}{8}\\
\\
-\dfrac{1+\Delta}{4a}=\left(1+\dfrac{81}{4}\right)\left(-\dfrac{1}{6}\right)=\dfrac{4+81}{4}\left(-\dfrac{1}{6}\right)=\dfrac{85}{24}\\
\end{cases} \]
Ricapitolando \[F\coord{0}{\dfrac{77}{4}} \]\[V\coord{0}{\dfrac{27}{8}} \]\[y=\dfrac{81}{24} \]
\begin{center}
	\includestandalone[width=.5\textwidth]{terzo/grafici/parabola3b}
	\captionof{figure}{Elementi della parabola}\label{fig:disegnoparabola3b}
\end{center}
\begin{esempiot}{Elementi di una parabola}{parabola3c}
	Trovare asse, fuoco, vertice e direttrice della parabola $y=\dfrac{1}{4}x^2$
\end{esempiot}
$a<0$ la parabola ha concavità rivolta verso il basso. I coefficienti sono\[\begin{cases}
a=\dfrac{1}{4}\\
b=0\\
c=0
\end{cases} \]
L'asse della parabola è\[x=-\dfrac{-b}{2a}=-\dfrac{0}{2\cdot\dfrac{1}{4}}=0 \]
Per trovare le coordinate del fuoco $F\coord{-\dfrac{b}{2a}}{\dfrac{1-\Delta}{4a}}$, del vertice $V\coord{-\dfrac{b}{2a}}{-\dfrac{\Delta}{4a}}$ e l'equazione della direttrice $y=-\dfrac{1+\Delta}{4a}$, bisogna conoscere il valore di $\Delta$ e di $4a$. Inizio con il calcolare \[\Delta=b^2-4a=0-0=0\]
\[4a=4\cdot\left(\dfrac{1}{4}\right)=1\]
Quindi \[\begin{cases}
\dfrac{1-\Delta}{4a}=\dfrac{1-0}{1}=1\\
\\
-\dfrac{\Delta}{4a}=-\dfrac{0}{1}=0\\
\\
-\dfrac{1+\Delta}{4a}=\dfrac{1+0}{1}=-1\\
\end{cases} \]
Ricapitolando \[F\coord{0}{1} \]\[V\coord{0}{0} \]\[y=-1 \]
\begin{center}
	\includestandalone[width=.5\textwidth]{terzo/grafici/parabola3c}
	\captionof{figure}{Elementi della parabola}\label{fig:disegnoparabola3c}
\end{center}
\section{Intersezioni}
\begin{esempiot}{Intersezione della parabola con gli assi}{parabola4}
	Trovare i punti di intersezione della parabola $y=3x^2+2x+5$ con gli assi.
\end{esempiot}
Intersezione asse $x$
\begin{align*}
	&\begin{cases}
		y=0\\y=3x^2+2x-5
	\end{cases}\\
	&3x^2+2x-5=0\\
	&x_{1.2}=\dfrac{-2\pm\sqrt{4+60}}{6}=\\
	&=\dfrac{-2\pm\sqrt{64}}{6}=\\
	&=\dfrac{-2\pm 8}{6}=\begin{cases}
		x_1=-\dfrac{10}{6}=-\dfrac{5}{3}\\
		x_2=\dfrac{6}{6}=1
	\end{cases}
\end{align*}
Intersezione asse $y$
\begin{align*}
&\begin{cases}
	x=0\\y=3x^2+2x-5
\end{cases}&\begin{cases}
x=0\\y=3\cdot0+2\cdot 0-5
\end{cases}\\&\begin{cases}
x=0\\y=-5
\end{cases}
\end{align*}
Quindi \[A\coord{-\dfrac{5}{3}}{0} \] \[B\coord{1}{0} \] \[C\coord{0}{-5} \]
\begin{center}
	\includestandalone[width=.5\textwidth]{terzo/grafici/parabola4}
	\captionof{figure}{Intersezioni con gli assi}\label{fig:disegnoparabola4}
\end{center}
\begin{esempiot}{Intersezione della parabola con gli assi}{parabola5}
	Trovare i punti di intersezione della parabola $y=x^2+2x+1$ con gli assi.
\end{esempiot}
Intersezione asse $x$
\begin{align*}
	&\begin{cases}
		y=0\\y=x^2+2x+1
	\end{cases}\\
	&x^2+2x+1=0\\
	&x_{1.2}=\dfrac{-2\pm\sqrt{4-4}}{2}=\\
	&=\dfrac{-2\pm\sqrt{0}}{2}=\\
	&=-1
\end{align*}
Intersezione asse $y$
\begin{align*}
	&\begin{cases}
		x=0\\y=x^2+2x+1
	\end{cases}&\begin{cases}
	x=0\\y=0+2\cdot 0+1
\end{cases}\\&\begin{cases}
x=0\\y=1
\end{cases}
\end{align*}
Quindi \[A\coord{-1}{0} \]  \[B\coord{0}{1} \]
\begin{center}
	\includestandalone[width=.5\textwidth]{terzo/grafici/parabola5}
	\captionof{figure}{Intersezioni con gli assi}\label{fig:disegnoparabola5}
\end{center}
\begin{esempiot}{Intersezione della parabola con gli assi}{parabola6}
	Trovare i punti di intersezione della parabola $y=-2x^2+3x-4$ con gli assi.
\end{esempiot}
Intersezione asse $x$
\begin{align*}
	&\begin{cases}
		y=0\\y=-2x^2+3x-4
	\end{cases}\\
	&-2x^2+3x-4=0\\
	&x_{1.2}=\dfrac{-3\pm\sqrt{9-32}}{-4}=\\
\end{align*}
La parabola non interseca l'asse.\par
Intersezione asse $y$
\begin{align*}
	&\begin{cases}
		x=0\\y=-2x^2+3x-4
	\end{cases}&\begin{cases}
	x=0\\y=-2\cdot 0+3\cdot 0-4
\end{cases}\\&\begin{cases}
x=0\\y=-4
\end{cases}
\end{align*}
Quindi \[A\coord{0}{-4} \] 
\begin{center}
	\includestandalone[width=.5\textwidth]{terzo/grafici/parabola6}
	\captionof{figure}{Intersezioni con gli assi}\label{fig:disegnoparabola6}
\end{center}
\section{Parabola per tre punti}
\begin{esempiot}{Parabola per tre punti}{parabola7}
	Trovare la parabola che passa per i punti $A\coord{3}{5}$, $B\coord{2}{3}$ e $C\coord{-1}{5}$
\end{esempiot}
Consideriamo le parabola $y=ax^2+bx+c$

Con il passaggio per i tre punti otteniamo
\begin{align*}
	&\begin{cases}
		9a+3b+c=5\\4a+2b+c=3\\a-b+c=5
	\end{cases}
		&&\begin{cases}
			c=5-9a-3b\\4a+2b+5-9a-3b=3\\a-b+5-9a-3b=5
		\end{cases}\\
		&\begin{cases}
			c=5-9a-3b\\-5a-b=-2\\-8a-4b=0
		\end{cases}
		&&\begin{cases}
			c=5-9a-3b\\2a+b=0\\5a+b=2
		\end{cases}\\
		&\begin{cases}
			c=5-9a+6a\\b=-2a\\3a=2
		\end{cases}
		&&\begin{cases}
			c=5-3a\\b=-2a\\3a=2
		\end{cases}\\
			&\begin{cases}
		a=\dfrac{2}{3}\\b=-\dfrac{4}{3}\\c=5-3\cdot\dfrac{2}{3}
			\end{cases}
			&&\begin{cases}
			a=\dfrac{2}{3}\\b=-\dfrac{4}{3}\\c=3
			\end{cases}\\
\end{align*}
otteniamo\[y=\dfrac{2}{3}x^2-\dfrac{4}{3}x+3 \]
\begin{center}
	\includestandalone[width=.5\textwidth]{terzo/grafici/parabola7}
	\captionof{figure}{Parabola per tre punti}\label{fig:disegnoparabola7}
\end{center}
\begin{esempiot}{Parabola per tre punti}{parabola8}
	Trovare la parabola che passa per i punti $A\coord{1}{0}$, $B\coord{3}{0}$ e $C\coord{0}{4}$
\end{esempiot}
Consideriamo le parabola $y=ax^2+bx+c$

Con il passaggio per i tre punti otteniamo
\begin{align*}
	&\begin{cases}
		a+b+c=0\\9a+3b+c=0\\c=4
	\end{cases}
	&&\begin{cases}
		a+b+4=0\\9a+3b+4=0\\c=4
	\end{cases}\\
	&\begin{cases}
		c=4\\a=-b-4\\9(-b-4)+3b+4=0
	\end{cases}
	&&\begin{cases}
			c=4\\a=-b-4\\-9b-36+3b+4=0
	\end{cases}\\
	&\begin{cases}
			c=4\\a=-b-4\\-6b-32=0
	\end{cases}
	&&\begin{cases}
		c=4\\a=-b-4\\6b=-32
	\end{cases}\\
	&\begin{cases}
		c=4\\a=+\dfrac{62}{6}-4\\b=-\dfrac{62}{6}
	\end{cases}
	&&\begin{cases}
		a=\dfrac{32-24}{6}=\dfrac{8}{6}=\dfrac{4}{3}\\b=-\dfrac{32}{6}=-\dfrac{16}{3}\\c=4
	\end{cases}\\
\end{align*}
Quindi\[y=\dfrac{4}{3}x^2-\dfrac{16}{3}x+4 \]
\begin{center}
	\includestandalone[width=.5\textwidth]{terzo/grafici/parabola8}
	\captionof{figure}{Parabola per tre punti}\label{fig:disegnoparabola8}
\end{center}
\begin{esempiot}{Parabola per tre punti}{parabola9}
	Trovare la parabola che passa per i punti $A\coord{1}{3}$, $B\coord{3}{0}$ e $C\coord{0}{0}$
\end{esempiot}
Consideriamo le parabola $y=ax^2+bx+c$

Con il passaggio per i tre punti otteniamo
\begin{align*}
&\begin{cases}
a+b+c=3\\9a+3b+c=0\\c=0
\end{cases}
&&\begin{cases}
a+b=3\\9a+3b=0\\c=0
\end{cases}\\
&\begin{cases}
c=0\\b=-3a\\a-3a=3
\end{cases}
&&\begin{cases}
c=0\\b=-3a\\a-3a=3
\end{cases}\\
&\begin{cases}
c=0\\-2a=3\\b=3a
\end{cases}
&&\begin{cases}
c=0\\a=-\dfrac{3}{2}\\b=\dfrac{9}{2}
\end{cases}
\end{align*}
\begin{center}
	\includestandalone[width=.5\textwidth]{terzo/grafici/parabola9}
	\captionof{figure}{Parabola per tre punti}\label{fig:disegnoparabola9}
\end{center}
Quindi \[y=-\dfrac{3}{2}x^2+ \dfrac{9}{2}x\]
\section{Intersezioni retta parabola}
\begin{esempiot}{Intersezioni retta parabola}{parabola10}
	Trovare i punti di intersezione fra la retta $y=3x-5$ e la parabola di equazione $y=3x^2+2x+5$
\end{esempiot}
Imposto il sistema
\begin{align*}
&\begin{cases}
y=3x^2+2x-5\\
y=3x-5
\end{cases}\\
&3x^2+2x-5=3x-5\\
&3x^2+2x-5-3x+5=0\\
&3x^2-x=0\\
&x(3x-1)=0\\
&x_1=0\\
&3x-1=0\\
&x_2=\dfrac{1}{3}\\
&\begin{cases}
x_1=0\\
y=3x-5
\end{cases}
&&\begin{cases}
x_1=0\\
y=-5
\end{cases}\\
&\begin{cases}
x_2=\dfrac{1}{3}\\
y=3x-5
\end{cases}
&\begin{cases}
x_2=\dfrac{1}{3}\\
\\
y=3\dfrac{1}{3}-5
\end{cases}
&\begin{cases}
x_2=\dfrac{1}{3}\\
\\
y=-4
\end{cases}
\end{align*}
%\begin{center}
%	\includestandalone[width=\textwidth]{terzo/grafici/parabola10}
%	\captionof{figure}{Intersezione retta parabola}\label{fig:disegnoparabola10}
%\end{center}
\begin{esempiot}{Intersezioni retta parabola}{parabola11}
	Trovare i punti di intersezione fra la retta $y=2x+7$ e la parabola di equazione $y=x^2+2x+6$
\end{esempiot}
Imposto il sistema
\begin{align*}
&\begin{cases}
y=x^2+2x+6\\
y=2x+7
\end{cases}\\
&x^2+2x+6=2x+7\\
&x^2+2x+6-2x-7=0\\
&x^2-1=0\\
&x_1=+1\\
&x_2=-1\\
&\begin{cases}
x_1=+1\\
y=2\cdot 1+7
\end{cases}
&&\begin{cases}
x_1=1\\
y=9
\end{cases}\\
&\begin{cases}
x_2=-1\\
y=2\cdot(-1)+7
\end{cases}
&&\begin{cases}
x_2=-1\\
y=5
\end{cases}
\end{align*}
\begin{center}
	\includestandalone[width=.5\textwidth]{terzo/grafici/parabola11}
	\captionof{figure}{Intersezioni retta parabola}\label{fig:disegnoparabola11}
\end{center}
\begin{esempiot}{Intersezioni retta parabola}{parabola12}
	Trovare i punti di intersezione fra la retta $y=x+3$ e la parabola di equazione $y=x^2+3x+5$
\end{esempiot}
Imposto il sistema
\begin{align*}
&\begin{cases}
y=x^2+3x+5\\
y=x+3
\end{cases}\\
&x^2+3x+5=x+3\\
&x^2+3x+5-x-3=0\\
&x^2+2x+2=0\\
&x_{1,2}=\dfrac{-2\pm\sqrt{4-4\cdot 1\cdot 2}}{2}
\end{align*}
Non ha soluzione.
\begin{center}
	\includestandalone[width=.5\textwidth]{terzo/grafici/parabola12}
	\captionof{figure}{Intersezioni retta parabola}\label{fig:disegnoparabola12}
\end{center}
\begin{esempiot}{Intersezioni retta parabola}{parabola13}
	Trovare i punti di intersezione fra la retta $y=x+1$ e la parabola di equazione $y=9x^2-5x+2$
\end{esempiot}
Imposto il sistema
\begin{align*}
&\begin{cases}
y=9x^2-5x+2\\
y=x+1
\end{cases}\\
&9x^2-5x+2=x+1\\
&9x^2-5x+2-x-1=0\\
&9x^2-6x+1=0\\
&x_{1,2}=\dfrac{+6\pm\sqrt{36-4\cdot 1\cdot 9}}{18}\\
&x_{1,2}=\dfrac{+6\pm\sqrt{36-36}}{18}\\
&=\dfrac{6}{18}\\
&=\dfrac{1}{3}
&\begin{cases}
x_1=\dfrac{1}{3}\\
y=\dfrac{1}{3}+1
\end{cases}
&\begin{cases}
x_1=\dfrac{1}{3}\\
y=\dfrac{1+3}{3}
\end{cases}
&\begin{cases}
x_1=\dfrac{1}{3}\\
y=\dfrac{4}{3}
\end{cases}
\end{align*}

\begin{center}
	\includestandalone[width=.5\textwidth]{terzo/grafici/parabola13}
	\captionof{figure}{Intersezioni retta parabola}\label{fig:disegnoparabola13}
\end{center}
 
%\begin{exercise}
%	Calcola il perimetro della che ha per vertici $A\coord{-3}{1}$, $B\coord{-1}{4}$, $C\coord{-6}{4}$ e  $D\coord{-8}{1}$
%	\tcblower
%	
%\end{exercise}
\chapter{Esercizi di riepilogo}
\tcbstartrecording
\section{Perimetro figure}
\begin{exercise}
	Calcola il perimetro del triangolo che ha per vertici $A\coord{-3}{2}$, $B\coord{3}{2}$ e $C\coord{0}{-3}$
	\tcblower
	Calcolo la distanza tra $AB$ come con l'\cref{exa:disuno} 
	quindi \[d(AB)=\abs{-3-3}=\abs{-6}=6\] Ora calcoliamo la distanza tra $BC$ come con \cref{exa:discinque} 
	\begin{align*}
		d(BC)=&\sqrt{(3)^2+(2+3)^2}\\
		=&\sqrt{34}
	\end{align*}
	Ora calcoliamo la distanza tra $AC$ come con \cref{exa:discinque} 
	\begin{align*}
		d(AC)=&\sqrt{(-3)^2+(2+3)^2}\\
		=&\sqrt{34}
	\end{align*}
	Conseguentemente il perimetro è
	\[2P=6+\sqrt{34}+\sqrt{34}=6+2\sqrt{34}\]
	\begin{center}
		\includestandalone[width=.5\textwidth]{terzo/grafici/retta_dis_11}
		\captionof{figure}{Perimetro triangolo}\label{fig:EsRieDistanza11}
	\end{center}
\end{exercise}
\begin{exercise}
	Calcola il perimetro della figura che ha per vertici i punti $A\coord{-3}{1}$, $B\coord{-1}{4}$, $C\coord{-6}{4}$ e  $D\coord{-8}{1}$
	\tcblower
	Calcoliamo la lunghezza tra $AB$ come con \cref{exa:discinque} 
	\begin{align*}
		d(AB)=&\sqrt{(-3+1)^2+(1-4)^2}\\
		=&\sqrt{4+9}\\
		=&\sqrt{13}\\
	\end{align*}
	Calcolo la distanza tra $CB$ come con l'\cref{exa:disuno} 
	quindi \[d(CB)=\abs{-1+6}=\abs{5}=5\]
	Ora calcoliamo la distanza tra $CD$ come con \cref{exa:discinque} 
	\begin{align*}
		d(CD)=&\sqrt{(-6+8)^2+(4-1)^2}\\
		=&\sqrt{4+9}\\
		=&\sqrt{13}
	\end{align*}
	Calcolo la distanza tra $DA$ come con l'\cref{exa:disuno} 
	quindi \[d(DA)=\abs{-3+8}=\abs{5}=5\]
	Conseguentemente il perimetro è
	\[2P=5+\sqrt{13}+5+\sqrt{13}=10+2\sqrt{13}\]
	\begin{center}
		\includestandalone[width=.5\textwidth]{terzo/grafici/retta_dis_12}
		\captionof{figure}{Perimetro}\label{fig:EsRieDistanza12}
	\end{center}
\end{exercise}
\section{Area Triangoli}
\begin{exercise}
	Calcola l'area del triangolo che ha per vertici $A\coord{1}{1}$, $B\coord{6}{1}$ e $C\coord{4}{3}$
	\tcblower
	Calcola l'area del triangolo che ha per vertici $A\coord{1}{1}$, $B\coord{6}{1}$ 
	e $C\coord{4}{3}$
	L'area del triangolo è data da \[A=\dfrac{b\cdot h}{2}\]
	Quindi devo conoscere la lunghezza della base $AB$ e dell'altezza $CH$.
	La base $AB$ è un segmento orizzontale quindi \[b=d(AB)=\left| x_A-x_B\right|=\left| 1-6\right|=\left|-5\right|=5   \]
	Per trovare l'altezza prima troviamo le coordinate del punto $H$. Dato che è sul segmento $AB$ il punto $H$ ha la stessa ordinata di $A$. Inoltre dato che è sul segmento perpendicolare $CH$, $H$ ha la stessa ascissa di $C$. Quindi,  $H\coord{4}{1}$. Abbiamo \[h=d(CH)=\left| y_C-y_H\right|=\left| 1-3\right|=\left|-2\right|=2   \]In conclusione: \[A=\dfrac{b\cdot h}{2}=\dfrac{5\cdot 2}{2}=5\]
	\begin{center}
		\includestandalone[width=.5\textwidth]{terzo/grafici/AreaTriangolo1}
		\captionof{figure}{Aera triangolo}\label{fig:AreaTriangolo1}
	\end{center}
\end{exercise}
\begin{exercise}
	Calcola l'area del triangolo che ha per vertici $A\coord{-3}{1}$, $B\coord{1}{1}$ e $C\coord{3}{4}$
	\tcblower
	Calcola l'area del triangolo che ha per vertici $A\coord{-3}{1}$, $B\coord{1}{1}$ e $C\coord{3}{4}$
	L'area del triangolo è data da \[A=\dfrac{b\cdot h}{2}\]
	Quindi devo conoscere la lunghezza della base $AB$ e dell'altezza $CH$.
	La base $AB$ è un segmento orizzontale quindi \[b=d(AB)=\left| x_A-x_B\right|=\left|-3-1\right|=\left|-4\right|=4   \]
	Per trovare l'altezza prima troviamo le coordinate del punto $H$. Dato che è esterno al segmento $AB$ il punto $H$ si trova sul prolungamento di $AB$ e ha la stessa ordinata di $A$. Inoltre dato che è sul segmento perpendicolare $CH$, $H$ ha la stessa ascissa di $C$. Quindi,  $H\coord{3}{1}$. Abbiamo \[h=d(CH)=\left| y_C-y_H\right|=\left| 4-1\right|=\left|3\right|=3   \]In conclusione: \[A=\dfrac{b\cdot h}{2}=\dfrac{4\cdot 3}{2}=6\]
	\begin{center}
		\includestandalone[width=.5\textwidth]{terzo/grafici/AreaTriangolo2}
		\captionof{figure}{Aera triangolo}\label{fig:AreaTriangolo2}
	\end{center}
\end{exercise}

\section{Disegnare rette}
\begin{exercise}
	Disegnare nel piano cartesiano la retta di equazione \[3x+y-1=0\]
	\tcblower
	Disegnare nel piano cartesiano la retta di equazione \[3x+y-1=0\]
	La retta è in forma implicita la trasformo in forma esplicita ottengo\[y=-3x+1\]
	\begin{center}
		\includestandalone[width=.5\textwidth]{terzo/grafici/retta_dis_21}
		\captionof{figure}{Disegnare rette}\label{fig:DiegnareRette1}
	\end{center}
\end{exercise}
\section{Retta per punto}
\begin{exercise}
	Data la retta $y=-x+2$ verificare se passa per i punti $P(2,0)$ e $Q(1,5)$
	\tcblower
	Data la retta $y=-x+2$ verificare se passa per i punti $P(2,0)$ e $Q(1,5)$
	
	Verifichiamo per $P$
	\begin{align*}
		y=&-x+2\\
		\intertext{Sostituisco le coordinate di $P$}
		0=&-1\cdot 2+2\\
		0=&-2+2\\
		0=&0\\
	\end{align*}
	
	Vero, la retta passa per $P$
	
		Verifichiamo per $Q$
	\begin{align*}
		y=&-x+2\\
		\intertext{Sostituisco le coordinate di $Q$}
		5=&-1\cdot 1+2\\
		5=&-1+2\\
		5=&1\\
	\end{align*}
	
	Falso, la retta non passa per $Q$
\end{exercise}
\begin{exercise}
	Data la retta $3x+2y-5=0$ verificare se passa per i punti $P(3,4)$ e $Q(1,1)$
	\tcblower
	Data la retta $3x+2y-5=0$ verificare se passa per i punti $P(3,4)$ e $Q(1,1)$
	
	Verifichiamo per $P$
	\begin{align*}
		3x+2y-5=0\\
		\intertext{Sostituisco le coordinate di $P$}
		3\cdot 3+2\cdot 4-5=&0\\
		9+8-5=&0\\
		12=&0\\
	\end{align*}
	
	Falso, la retta non passa per $P$
	
	Verifichiamo per $Q$
	\begin{align*}
		3x+2y-5=0\\
		\intertext{Sostituisco le coordinate di $P$}
		3\cdot 1+2\cdot 1-5=&0\\
		5-5=&0\\
		0=&0\\
	\end{align*}
	
	Vero, la retta passa per $Q$
\end{exercise}
\begin{exercise}
	Data la retta $y=-2x+7$ verificare se passa per i punti $P(3,1)$ e $Q(2,3)$
	\tcblower
	Data la retta $y=-2x+7$ verificare se passa per i punti $P(3,1)$ e $Q(2,3)$
	
	Verifichiamo per $P$
	\begin{align*}
	y=&-2x+7\\
		\intertext{Sostituisco le coordinate di $P$}
		1=&-2\cdot 3 +7\\
		1=&-6 +7\\
		1=&1
	\end{align*}
	
	Vero, la retta  passa per $P$
	
	Verifichiamo per $Q$
\begin{align*}
	y=&-2x+7\\
	\intertext{Sostituisco le coordinate di $Q$}
	3=&-2\cdot 2 +7\\
	3=&-4 +7\\
	3=&3
\end{align*}
	
	Vero, la retta passa per $Q$
\end{exercise}
\begin{exercise}[no solution]
	Data la retta $4x-2y-1$ verificare se passa per il punto $P(0,1)$
\end{exercise}
\begin{exercise}[no solution]
	Data la retta $2x+2y+2=0$ verificare se passa per il punto $P(-3,2)$
\end{exercise}
\begin{exercise}[no solution]
	Data la retta $y=3x+1$ verificare se passa per il punto $P(1,4)$
\end{exercise}
\begin{exercise}[no solution]
	Data la retta $y=-3x+1$ verificare se passa per il punto $P(1,4)$
\end{exercise}
\section{Rette parallele e perpendicolari}
\begin{exercise}
	La retta $7x+6y+4=0$ incontra l'asse $y$ nel punto $A$. Trovare la retta perpendicolare e la retta parallela alla retta $8x+5y+1=0$ che passano per $A$
	\tcblower
	La retta $7x+6y+4=0$ incontra l'asse $y$ nel punto $A$. Trovare la retta perpendicolare e la retta parallela alla retta $8x+5y+1=0$ che passano per $A$.
	Troviamo le coordinate del punto $A$ scrivo la retta  $7x+6y+4=0$ in forma esplicita
	\begin{align*}
		6y=&-7x-4\\
		y=&-\dfrac{7}{6}x-\dfrac{4}{6}\\
		y=&-\dfrac{7}{6}x-\dfrac{2}{3}\\
	\end{align*}
	Quindi $q=-\dfrac{2}{3}$ per cui $A\coord{0}{-\dfrac{2}{3}}$. Per trovare la retta perpendicolare a $8x+5y+1=0$ devo calcolarne il coefficiente angolare.
	Scrivo la retta in forma esplicita.
	\begin{align*}
		5y=&-8x-1\\
		y=&-\dfrac{8}{5}x-\dfrac{1}{5}\\
	\end{align*}
	$m=-\dfrac{8}{5}$ utilizzando la formula \[m_1\cdot m_2=-1\] Otteniamo
	\begin{align*}
		-\dfrac{8}{5}\cdot m_2=&-1\\
		\dfrac{8}{5}\cdot m_2=&1\\
		m_2=&\dfrac{5}{8}\\
	\end{align*}
	Utilizzando l'equazione del fascio di rette, il valore di $m_2$ trovato e le coordinate di $A$ Otteniamo:
	\begin{align*}
		y+\dfrac{2}{5}=&\dfrac{5}{8}x\\
		y=&\dfrac{5}{8}x-\dfrac{2}{5}
	\end{align*}
	Cioè l'equazione della retta perpendicolare cercata.
	
	Due rette sono parallele se hanno lo stesso coefficiente angolare quindi \[m_1=m_2 \]
	quindi $m_2=-\dfrac{8}{5}$ 
	
	Utilizzando l'equazione del fascio di rette, il valore di $m_2$ trovato e le coordinate di $A$ Otteniamo:
	\begin{align*}
		y+\dfrac{2}{5}=&-\dfrac{8}{5}x\\
		y=&-\dfrac{8}{5}x-\dfrac{2}{5}
	\end{align*}
	Cioè l'equazione della retta parallela cercata.		
	\begin{center}
		\includestandalone[width=.5\textwidth]{terzo/grafici/retta_dis_13}
		%\captionof{figure}{Grafico}\label{fig:EsRiedistanza13}
	\end{center}
\end{exercise}
\begin{exercise}
	Data la retta $r$ di equazione $y=7x+6$, trovarne le intersezioni con gli assi. Indicato con $A$ il punto di intersezione di $r$ con l'asse $x$ e con $B$ il punto di intersezione con $r$ l'asse $y$. Trovare la retta $s$ perpendicolare a $r$ che passa $A$.
	Trovare la retta $t$ parallela a $s$ che passa per $B$. Disegnare la rette.
	\tcblower
	Data la retta $r$ di equazione $y=7x+6$, trovarne le intersezioni con gli assi. Indicato con $A$ il punto di intersezione di $r$ con l'asse $x$ e con $B$ il punto di intersezione con $r$ l'asse $y$. Trovare la retta $s$ perpendicolare a $r$ che passa $A$.
	Trovare la retta $t$ parallela a $s$ che passa per $B$. Disegnare la rette.
	
	Intersezione con l'asse $x$ di $y=7x+6$
	\begin{align*}
		7x+6=&0\\
		7x=&-6\\
		x=&-\dfrac{6}{7}
	\end{align*}
	$A\coord{-\dfrac{6}{7}}{0}$
	
	Intersezione con l'asse $y$
	$A\coord{0}{6}$
	
	$m=7$ utilizzando la formula \[m_1\cdot m_2=-1\] Otteniamo
	\begin{align*}
		7\cdot m_2=&-1\\
		m_2=&-\dfrac{1}{7}\\
	\end{align*}
	
	Utilizzando l'equazione del fascio di rette, il valore di $m_2$ trovato e le coordinate di $A$ Otteniamo:
	\begin{align*}
		y=&-\dfrac{1}{7}(x+\dfrac{6}{7})\\
		y=&-\dfrac{1}{7}x-\dfrac{6}{49}
	\end{align*}
	Cioè l'equazione della retta perpendicolare cercata.
	
	Due rette sono parallele se hanno lo stesso coefficiente angolare quindi \[m_1=m_2 \]
	quindi $m_2=-\dfrac{1}{7}$ 
	
	Utilizzando l'equazione del fascio di rette, il valore di $m_2$ trovato e le coordinate di $B$ Otteniamo:
	\begin{align*}
		y-6=&-\dfrac{1}{7}x\\
		y=&-\dfrac{1}{7}x+6\\
	\end{align*}
	Cioè l'equazione della retta parallela cercata.	
	\begin{center}
		\includestandalone[width=.5\textwidth]{terzo/grafici/retta_dis_14}
		%\captionof{figure}{Grafico}\label{fig:EsRiedistanza13}
	\end{center} 	
\end{exercise}
\begin{exercise}
	Trovare la retta che passa per $A\coord{7}{6}$, parallela alla retta che passa per $B\coord{-13}{8}$ e $C\coord{5}{1}$
	\tcblower
	Trovare la retta che passa per $A\coord{7}{6}$ , parallela alla retta che passa per $B\coord{-13}{8}$ e $C\coord{5}{1}$
	
	Troviamo il coefficiente angolare della retta che passa per $B$ e $C$
	\begin{align*}
		y-8=&(x+13)\\
		1-8=&(5+13)\\
		-7=&18m\\
		m=&-\dfrac{7}{18}
	\end{align*}
	
	Due rette sono parallele se hanno lo stesso coefficiente angolare quindi \[m_1=m_2 \]
	quindi $m_2=-\dfrac{7}{18}$ 
	
	Utilizzando l'equazione del fascio di rette, il valore di $m_2$ trovato e le coordinate di $A$ Otteniamo:
	\begin{align*}
		y-6=&-\dfrac{7}{18}(x-7)\\
		y=&-\dfrac{7}{18}x+\dfrac{49}{18}+6\\
		y=&-\dfrac{7}{18}x+\dfrac{157}{18}\\
	\end{align*}
	Cioè l'equazione della retta parallela cercata.	
	\begin{center}
		\includestandalone[width=.5\textwidth]{terzo/grafici/retta_dis_15}
		%\captionof{figure}{Grafico}\label{fig:EsRiedistanza13}
	\end{center}
\end{exercise}	
\begin{exercise}
	Data la retta $-3x+14y-13=0$, scrivi l'equazione della retta parallela e della retta perpendicolare che passano per  $A\coord{-5}{-2}$
	\tcblower
	Data la retta $-3x+14y-13=0$, scrivi l'equazione della retta parallela e della retta perpendicolare che passano per  $A\coord{-5}{-2}$
	
	Scrivo la retta in forma esplicita
	\begin{align*}
		-3x+14y-13=&0\\
		14y=&+3x+13\\
		y=&\dfrac{3}{13}x+\dfrac{13}{14}
	\end{align*}
	$m=\dfrac{3}{14}$ utilizzando la formula \[m_1\cdot m_2=-1\] Otteniamo
	\begin{align*}
		\dfrac{3}{14}\cdot m_2=&-1\\
		m_2=&-\dfrac{14}{3}\\
	\end{align*}
	Utilizzando l'equazione del fascio di rette, il valore di $m_2$ trovato e le coordinate di $A$ Otteniamo:
	\begin{align*}
		y+2=&-\dfrac{14}{3}(x+5)\\
		y=&-\dfrac{14}{3}x-\dfrac{70}{3}-2\\
		y=&-\dfrac{14}{3}x-\dfrac{76}{3}\\
	\end{align*}
	Cioè l'equazione della retta perpendicolare cercata.
	
	Due rette sono parallele se hanno lo stesso coefficiente angolare quindi \[m_1=m_2 \]
	quindi $m_2=\dfrac{3}{14}$ 
	
	Utilizzando l'equazione del fascio di rette, il valore di $m_2$ trovato e le coordinate di $A$ Otteniamo:
	\begin{align*}
		y+2=&\dfrac{3}{14}(x+5)\\
		y=&\dfrac{3}{14}x+\dfrac{15}{14}-2\\
		y=&-\dfrac{3}{14}x-\dfrac{13}{14}\\
	\end{align*}
	Cioè l'equazione della retta parallela cercata.
	
	\begin{center}
		\includestandalone[width=.5\textwidth]{terzo/grafici/retta_dis_16}
		%\captionof{figure}{Grafico}\label{fig:EsRiedistanza13}
	\end{center}
\end{exercise}

\section{Retta intersezioni}
\begin{exercise}
	Dati i punti  $A\coord{-13}{-12}$, $B\coord{-15}{-8}$,trovare l'equazione della retta che passa per questi punti. Trovare le coordinate dei punti di intersezione con gli assi della retta trovata. Disegnare la retta.
	\tcblower
	Dati i punti  $A\coord{-13}{-12}$, $B\coord{-15}{-8}$,trovare l'equazione della retta che passa per questi punti. Trovare le coordinate dei punti di intersezione con gli assi della retta trovata. Disegnare la retta.
	
	Consideriamo l'equazione  $\dfrac{y-y_1}{y_2-y_1}=\dfrac{x-x_1}{x_2-x_1}$
	\begin{align*}
		\intertext{Passaggio per  $A\coord{-13}{-12}$, $B\coord{-15}{-8}$}
		\dfrac{y+12}{-8+12}=&\dfrac{x+13}{-15+13}\\
		\dfrac{y+12}{4}=&\dfrac{x+13}{-2}\\
		-2y-24=&4x+52\\
		4x+2y+76=&0\\
		2x+y+38=&0\\
	\end{align*}
	Otteniamo la retta cercata.
	Intersezione con l'asse $x$ di $2x+y+38=0$
	\begin{align*}
		2x+38=&0\\
		2x=&-38\\
		x=&-19
	\end{align*}
	$A\coord{-19}{0}$
	
	Intersezione con l'asse $y$
	$B\coord{0}{-38}$
	\begin{center}
		\includestandalone[width=.6\textwidth]{terzo/grafici/retta_dis_17}
		%\captionof{figure}{Grafico}\label{fig:EsRiedistanza13}
	\end{center}
\end{exercise}
\begin{exercise}
	Date le rette $y=3x+3$ e $y=-2x-2$ trovare, se esiste, il loro punto di intersezione $A$.
	\tcblower
	Date le rette $y=3x+3$ e $y=-2x-2$ trovare, se esiste, il loro punto di intersezione $A$.
	\begin{align*}
		&\begin{cases}
			y=3x+3\\
			y=-2x-2
		\end{cases}
		&&\begin{cases}
			y=3x+3\\
			3x+5=-2x-2
		\end{cases}\\
		&\begin{cases}
			y=3x+3\\
			5x=-3-2
		\end{cases}
		&&\begin{cases}
			y=3x+3\\
			5x=-5
		\end{cases}\\
		&\begin{cases}
			y=3(-1)+3\\
			x=-1
		\end{cases}
		&&\begin{cases}
			y=0\\
			x=-1
		\end{cases}
	\end{align*}
	$A\coord{-1}{0}$
	\begin{center}
		\includestandalone[width=.6\textwidth]{terzo/grafici/retta_dis_18}
		%\captionof{figure}{Grafico}\label{fig:EsRiedistanza13}
	\end{center}
\end{exercise}
\begin{exercise}
	Date le rette $y=-3x+1$ e $x+y-3=0$ trovare,se esiste, il loro punto di intersezione $A$. 
	\tcblower
	Date le rette $y=-3x+1$ e $x+y-3=0$ trovare,se esiste, il loro punto di intersezione $A$.
	\begin{align*}
		&\begin{cases}
			x+y-3=0\\
			y=-3x+1
		\end{cases}
		&&\begin{cases}
			x-3x+1-3=0\\
			y=-3x+1
		\end{cases}\\
		&\begin{cases}
			-2x-2=0\\
			y=-3x+1
		\end{cases}
		&&\begin{cases}
			y=-1\\
			y=-3(-1)+1
		\end{cases}\\
		&\begin{cases}
			y=4\\
			x=-1
		\end{cases}
	\end{align*}
	$A\coord{-1}{4}$
	\begin{center}
		\includestandalone[width=.6\textwidth]{terzo/grafici/retta_dis_19}
		%\captionof{figure}{Grafico}\label{fig:EsRiedistanza13}
	\end{center}
\end{exercise}
\begin{exercise}
	Date le rette $y=-2x+5$ e $y=x+1$ trovare,se esiste, il loro punto di intersezione $A$. Trovare l'equazione della parallela a $y=5x+4$ che passa per il punto $A$.
	\tcblower
	Date le rette $y=-2x+5$ e $y=x+1$ trovare,se esiste, il loro punto di intersezione $A$. Trovare l'equazione della parallela a $y=5x+4$ che passa per il punto $A$.
	\begin{align*}
		&\begin{cases}
			y=-2x+5\\
			y=x+1
		\end{cases}
		&&\begin{cases}
			-2x+5=x+1\\
			y=x+1
		\end{cases}\\
		&\begin{cases}
			-3x=-4\\
			y=x+1
		\end{cases}
		&&\begin{cases}
			y=\dfrac{4}{3}\\[.5em]
			y=\dfrac{4}{3}+1
		\end{cases}\\
		&\begin{cases}
			y=\dfrac{4}{3}\\[.5em]
			y=\dfrac{7}{3}
		\end{cases}
	\end{align*}
	$A\coord{\dfrac{4}{3}}{\dfrac{7}{3}}$
	Due rette sono parallele se hanno lo stesso coefficiente angolare quindi \[m_1=m_2 \]
	quindi $m_2=5$
	
	Utilizzando l'equazione del fascio di rette, il valore di $m_2$ trovato e le coordinate di $A$ Otteniamo:
	\begin{align*}
		y-\dfrac{7}{3}=&5(x-\dfrac{4}{3})\\
		y=&5x-\dfrac{20}{3}+\dfrac{7}{3}\\
		y=&5x-\dfrac{13}{3}\\
	\end{align*}
	Cioè l'equazione della retta parallela cercata.
	\begin{center}
		\includestandalone[width=.6\textwidth]{terzo/grafici/retta_dis_20}
		%\captionof{figure}{Grafico}\label{fig:EsRiedistanza13}
	\end{center}
	
\end{exercise}

\tcbstoprecording
\newpage
\section{Soluzioni esercizi di riepilogo}
\tcbinputrecords							

\backmatter
\begin{appendices}
	\input{../Mod_base/MezziUsati}
\end{appendices}

\addcontentsline{toc}{chapter}{\indexname}
\printindex
\end{document}
