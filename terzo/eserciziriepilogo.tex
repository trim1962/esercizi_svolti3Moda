%\begin{exercise}
%	Calcola il perimetro della che ha per vertici $A\coord{-3}{1}$, $B\coord{-1}{4}$, $C\coord{-6}{4}$ e  $D\coord{-8}{1}$
%	\tcblower
%	
%\end{exercise}
\chapter{Esercizi di riepilogo geometria analitica}
\section{Perimetro figure}
\tcbstartrecording
\begin{exercise}
Calcola il perimetro del triangolo che ha per vertici $A\coord{-3}{2}$, $B\coord{3}{2}$ e $C\coord{0}{-3}$
\tcblower
Calcolo la distanza tra $AB$ come con l'\cref{exa:disuno} 
quindi \[d(AB)=\abs{-3-3}=\abs{-6}=6\] Ora calcoliamo la distanza tra $BC$ come con \cref{exa:discinque} 
\begin{align*}
d(BC)=&\sqrt{(3)^2+(2+3)^2}\\
=&\sqrt{34}
\end{align*}
Ora calcoliamo la distanza tra $AC$ come con \cref{exa:discinque} 
\begin{align*}
d(AC)=&\sqrt{(-3)^2+(2+3)^2}\\
=&\sqrt{34}
\end{align*}
Conseguentemente il perimetro è
\[2P=6+\sqrt{34}+\sqrt{34}=6+2\sqrt{34}\]
\begin{center}
	\includestandalone[width=.5\textwidth]{terzo/grafici/retta_dis_11}
	\captionof{figure}{Perimetro triangolo}\label{fig:EsRieDistanza11}
\end{center}
\end{exercise}
\begin{exercise}
	Calcola il perimetro della che ha per vertici $A\coord{-3}{1}$, $B\coord{-1}{4}$, $C\coord{-6}{4}$ e  $D\coord{-8}{1}$
	\tcblower
	Calcoliamo la lunghezza tra $AB$ come con \cref{exa:discinque} 
	\begin{align*}
		d(AB)=&\sqrt{(-3+1)^2+(1-4)^2}\\
		=&\sqrt{4+9}\\
		=&\sqrt{13}\\
	\end{align*}
	Calcolo la distanza tra $CB$ come con l'\cref{exa:disuno} 
	quindi \[d(CB)=\abs{-1+6}=\abs{5}=5\]
	Ora calcoliamo la distanza tra $CD$ come con \cref{exa:discinque} 
	\begin{align*}
		d(CD)=&\sqrt{(-6+8)^2+(4-1)^2}\\
			=&\sqrt{4+9}\\
		=&\sqrt{13}
	\end{align*}
	Calcolo la distanza tra $DA$ come con l'\cref{exa:disuno} 
	quindi \[d(DA)=\abs{-3+8}=\abs{5}=5\]
	Conseguentemente il perimetro è
	\[2P=5+\sqrt{13}+5+\sqrt{13}=10+2\sqrt{13}\]
	\begin{center}
		\includestandalone[width=.5\textwidth]{terzo/grafici/retta_dis_12}
		\captionof{figure}{Perimetro}\label{fig:EsRieDistanza12}
	\end{center}
\end{exercise}
\section{Rette parallele e perpendicolari}
\begin{exercise}
	La retta $7x+6y+4=0$ incontra l'asse $y$ nel punto $A$. Trovare la retta perpendicolare e la retta parallela alla retta $8x+5y+1=0$ che passano per $A$
	\tcblower
	La retta $7x+6y+4=0$ incontra l'asse $y$ nel punto $A$. Trovare la retta perpendicolare e la retta parallela alla retta $8x+5y+1=0$ che passano per $A$.
	Troviamo le coordinate del punto $A$ scrivo la retta  $7x+6y+4=0$ in forma esplicita
	\begin{align*}
	6y=&-7x-4\\
	y=&-\dfrac{7}{6}x-\dfrac{4}{6}\\
	y=&-\dfrac{7}{6}x-\dfrac{2}{3}\\
	\end{align*}
	Quindi $q=-\dfrac{2}{3}$ per cui $A\coord{0}{-\dfrac{2}{3}}$. Per trovare la retta perpendicolare a $8x+5y+1=0$ devo calcolarne il coefficiente angolare.
	Scrivo la retta in forma esplicita.
		\begin{align*}
		5y=&-8x-1\\
		y=&-\dfrac{8}{5}x-\dfrac{1}{5}\\
		\end{align*}
	$m=-\dfrac{8}{5}$ utilizzando la formula \[m_1\cdot m_2=-1\] Otteniamo
	\begin{align*}
-\dfrac{8}{5}\cdot m_2=&-1\\
\dfrac{8}{5}\cdot m_2=&1\\
 m_2=&\dfrac{5}{8}\\
	\end{align*}
	Utilizzando l'equazione del fascio di rette, il valore di $m_2$ trovato e le coordinate di $A$ Otteniamo:
	\begin{align*}
	y+\dfrac{2}{5}=&\dfrac{5}{8}x\\
		y=&\dfrac{5}{8}x-\dfrac{2}{5}
	\end{align*}
	Cioè l'equazione della retta perpendicolare cercata.
	
	Due rette sono parallele se hanno lo stesso coefficiente angolare quindi \[m_1=m_2 \]
	quindi $m_2=-\dfrac{8}{5}$ 
	
	Utilizzando l'equazione del fascio di rette, il valore di $m_2$ trovato e le coordinate di $A$ Otteniamo:
		\begin{align*}
		y+\dfrac{2}{5}=&-\dfrac{8}{5}x\\
		y=&-\dfrac{8}{5}x-\dfrac{2}{5}
		\end{align*}
	Cioè l'equazione della retta parallela cercata.		
	\begin{center}
\includestandalone[width=.5\textwidth]{terzo/grafici/retta_dis_13}
%\captionof{figure}{Grafico}\label{fig:EsRiedistanza13}
\end{center}
\end{exercise}
\begin{exercise}
Data la retta $r$ di equazione $y=7x+6$, trovarne le intersezioni con gli assi. Indicato con $A$ il punto di intersezione di $r$ con l'asse $x$ e con $B$ il punto di intersezione con $r$ l'asse $y$. Trovare la retta $s$ perpendicolare a $r$ che passa $A$.
Trovare la retta $t$ parallela a $s$ che passa per $B$. Disegnare la rette.
\tcblower
Data la retta $r$ di equazione $y=7x+6$, trovarne le intersezioni con gli assi. Indicato con $A$ il punto di intersezione di $r$ con l'asse $x$ e con $B$ il punto di intersezione con $r$ l'asse $y$. Trovare la retta $s$ perpendicolare a $r$ che passa $A$.
Trovare la retta $t$ parallela a $s$ che passa per $B$. Disegnare la rette.
	
Intersezione con l'asse $x$ di $y=7x+6$
\begin{align*}
7x+6=&0\\
7x=&-6\\
x=&-\dfrac{6}{7}
\end{align*}
 $A\coord{-\dfrac{6}{7}}{0}$
 
 Intersezione con l'asse $y$
  $A\coord{0}{6}$
 
 	$m=7$ utilizzando la formula \[m_1\cdot m_2=-1\] Otteniamo
 	\begin{align*}
 	7\cdot m_2=&-1\\
 	m_2=&-\dfrac{1}{7}\\
 \end{align*}
 
 	Utilizzando l'equazione del fascio di rette, il valore di $m_2$ trovato e le coordinate di $A$ Otteniamo:
 	\begin{align*}
 	y=&-\dfrac{1}{7}(x+\dfrac{6}{7})\\
 	y=&-\dfrac{1}{7}x-\dfrac{6}{49}
 	\end{align*}
 	Cioè l'equazione della retta perpendicolare cercata.
 	
	Due rette sono parallele se hanno lo stesso coefficiente angolare quindi \[m_1=m_2 \]
	quindi $m_2=-\dfrac{1}{7}$ 
	
	Utilizzando l'equazione del fascio di rette, il valore di $m_2$ trovato e le coordinate di $B$ Otteniamo:
	\begin{align*}
	y-6=&-\dfrac{1}{7}x\\
	y=&-\dfrac{1}{7}x+6\\
	\end{align*}
	Cioè l'equazione della retta parallela cercata.	
		\begin{center}
			\includestandalone[width=.5\textwidth]{terzo/grafici/retta_dis_14}
			%\captionof{figure}{Grafico}\label{fig:EsRiedistanza13}
		\end{center} 	
\end{exercise}
\begin{exercise}
	Trovare la retta che passa per $A\coord{7}{6}$, parallela alla retta che passa per $B\coord{-13}{8}$ e $C\coord{5}{1}$
	\tcblower
	Trovare la retta che passa per $A\coord{7}{6}$ , parallela alla retta che passa per $B\coord{-13}{8}$ e $C\coord{5}{1}$
	
	Troviamo il coefficiente angolare della retta che passa per $B$ e $C$
	\begin{align*}
		y-8=&(x+13)\\
		1-8=&(5+13)\\
		-7=&18m\\
		m=&-\dfrac{7}{18}
	\end{align*}
	
		Due rette sono parallele se hanno lo stesso coefficiente angolare quindi \[m_1=m_2 \]
		quindi $m_2=-\dfrac{7}{18}$ 
		
		Utilizzando l'equazione del fascio di rette, il valore di $m_2$ trovato e le coordinate di $A$ Otteniamo:
		\begin{align*}
			y-6=&-\dfrac{7}{18}(x-7)\\
			y=&-\dfrac{7}{18}x+\dfrac{49}{18}+6\\
			y=&-\dfrac{7}{18}x+\dfrac{157}{18}\\
		\end{align*}
		Cioè l'equazione della retta parallela cercata.	
			\begin{center}
				\includestandalone[width=.5\textwidth]{terzo/grafici/retta_dis_15}
				%\captionof{figure}{Grafico}\label{fig:EsRiedistanza13}
			\end{center}
\end{exercise}	
\begin{exercise}
	Data la retta $-3x+14y-13=0$, scrivi l'equazione della retta parallela e della retta perpendicolare che passano per  $A\coord{-5}{-2}$
	\tcblower
	Data la retta $-3x+14y-13=0$, scrivi l'equazione della retta parallela e della retta perpendicolare che passano per  $A\coord{-5}{-2}$
	
	Scrivo la retta in forma esplicita
	\begin{align*}
		-3x+14y-13=&0\\
		14y=&+3x+13\\
		y=&\dfrac{3}{13}x+\dfrac{13}{14}
	\end{align*}
	 	$m=\dfrac{3}{14}$ utilizzando la formula \[m_1\cdot m_2=-1\] Otteniamo
	 	\begin{align*}
	 		\dfrac{3}{14}\cdot m_2=&-1\\
	 		m_2=&-\dfrac{14}{3}\\
	 	\end{align*}
	 		Utilizzando l'equazione del fascio di rette, il valore di $m_2$ trovato e le coordinate di $A$ Otteniamo:
	 		\begin{align*}
	 			y+2=&-\dfrac{14}{3}(x+5)\\
	 			y=&-\dfrac{14}{3}x-\dfrac{70}{3}-2\\
	 			y=&-\dfrac{14}{3}x-\dfrac{76}{3}\\
	 		\end{align*}
	 		Cioè l'equazione della retta perpendicolare cercata.
	 		
	 		Due rette sono parallele se hanno lo stesso coefficiente angolare quindi \[m_1=m_2 \]
	 		quindi $m_2=\dfrac{3}{14}$ 
	 		
	 		Utilizzando l'equazione del fascio di rette, il valore di $m_2$ trovato e le coordinate di $A$ Otteniamo:
	 		\begin{align*}
	 			y+2=&\dfrac{3}{14}(x+5)\\
	 			y=&\dfrac{3}{14}x+\dfrac{15}{14}-2\\
	 			y=&-\dfrac{3}{14}x-\dfrac{13}{14}\\
	 		\end{align*}
	 		Cioè l'equazione della retta parallela cercata.
	 		
	 			\begin{center}
	 				\includestandalone[width=.5\textwidth]{terzo/grafici/retta_dis_16}
	 				%\captionof{figure}{Grafico}\label{fig:EsRiedistanza13}
	 			\end{center}
\end{exercise}
\section{Retta intersezioni}
\begin{exercise}
	Dati i punti  $A\coord{-13}{-12}$, $B\coord{-15}{-8}$,trovare l'equazione della retta che passa per questi punti. Trovare le coordinate dei punti di intersezione con gli assi della retta trovata. Disegnare la retta.
	\tcblower
	Dati i punti  $A\coord{-13}{-12}$, $B\coord{-15}{-8}$,trovare l'equazione della retta che passa per questi punti. Trovare le coordinate dei punti di intersezione con gli assi della retta trovata. Disegnare la retta.
	
	Consideriamo l'equazione  $\dfrac{y-y_1}{y_2-y_1}=\dfrac{x-x_1}{x_2-x_1}$
	\begin{align*}
		\intertext{Passaggio per  $A\coord{-13}{-12}$, $B\coord{-15}{-8}$}
		\dfrac{y+12}{-8+12}=&\dfrac{x+13}{-15+13}\\
		\dfrac{y+12}{4}=&\dfrac{x+13}{-2}\\
		-2y-24=&4x+52\\
		4x+2y+76=&0\\
		2x+y+38=&0\\
	\end{align*}
	Otteniamo la retta cercata.
	Intersezione con l'asse $x$ di $2x+y+38=0$
	\begin{align*}
		2x+38=&0\\
		2x=&-38\\
		x=&-19
	\end{align*}
	$A\coord{-19}{0}$
	
	Intersezione con l'asse $y$
	$B\coord{0}{-38}$
		\begin{center}
			\includestandalone[width=.6\textwidth]{terzo/grafici/retta_dis_17}
			%\captionof{figure}{Grafico}\label{fig:EsRiedistanza13}
		\end{center}
\end{exercise}
\begin{exercise}
Date le rette $y=3x+3$ e $y=-2x-2$ trovare, se esiste, il loro punto di intersezione $A$.
\tcblower
Date le rette $y=3x+3$ e $y=-2x-2$ trovare, se esiste, il loro punto di intersezione $A$.
\begin{align*}
&\begin{cases}
y=3x+3\\
y=-2x-2
\end{cases}
&&\begin{cases}
y=3x+3\\
3x+5=-2x-2
\end{cases}\\
&\begin{cases}
y=3x+3\\
5x=-3-2
\end{cases}
&&\begin{cases}
y=3x+3\\
5x=-5
\end{cases}\\
&\begin{cases}
y=3(-1)+3\\
x=-1
\end{cases}
&&\begin{cases}
y=0\\
x=-1
\end{cases}
\end{align*}
	$A\coord{-1}{0}$
		\begin{center}
		\includestandalone[width=.6\textwidth]{terzo/grafici/retta_dis_18}
			%\captionof{figure}{Grafico}\label{fig:EsRiedistanza13}
		\end{center}
\end{exercise}
\begin{exercise}
Date le rette $y=-3x+1$ e $x+y-3=0$ trovare,se esiste, il loro punto di intersezione $A$. 
\tcblower
Date le rette $y=-3x+1$ e $x+y-3=0$ trovare,se esiste, il loro punto di intersezione $A$.
\begin{align*}
&\begin{cases}
x+y-3=0\\
y=-3x+1
\end{cases}
&&\begin{cases}
x-3x+1-3=0\\
y=-3x+1
\end{cases}\\
&\begin{cases}
-2x-2=0\\
y=-3x+1
\end{cases}
&&\begin{cases}
y=-1\\
y=-3(-1)+1
\end{cases}\\
&\begin{cases}
y=4\\
x=-1
\end{cases}
\end{align*}
$A\coord{-1}{4}$
\begin{center}
	\includestandalone[width=.6\textwidth]{terzo/grafici/retta_dis_19}
	%\captionof{figure}{Grafico}\label{fig:EsRiedistanza13}
\end{center}
\end{exercise}
\begin{exercise}
	Date le rette $y=-2x+5$ e $y=x+1$ trovare,se esiste, il loro punto di intersezione $A$. Trovare l'equazione della parallela a $y=5x+4$ che passa per il punto $A$.
	\tcblower
	Date le rette $y=-2x+5$ e $y=x+1$ trovare,se esiste, il loro punto di intersezione $A$. Trovare l'equazione della parallela a $y=5x+4$ che passa per il punto $A$.
	\begin{align*}
	&\begin{cases}
	y=-2x+5\\
	y=x+1
	\end{cases}
	&&\begin{cases}
	-2x+5=x+1\\
	y=x+1
	\end{cases}\\
	&\begin{cases}
	-3x=-4\\
	y=x+1
	\end{cases}
	&&\begin{cases}
	y=\dfrac{4}{3}\\[.5em]
	y=\dfrac{4}{3}+1
	\end{cases}\\
	&\begin{cases}
y=\dfrac{4}{3}\\[.5em]
y=\dfrac{7}{3}
	\end{cases}
	\end{align*}
	$A\coord{\dfrac{4}{3}}{\dfrac{7}{3}}$
	Due rette sono parallele se hanno lo stesso coefficiente angolare quindi \[m_1=m_2 \]
	quindi $m_2=5$
	
	Utilizzando l'equazione del fascio di rette, il valore di $m_2$ trovato e le coordinate di $A$ Otteniamo:
	\begin{align*}
	y-\dfrac{7}{3}=&5(x-\dfrac{4}{3})\\
	y=&5x-\dfrac{20}{3}+\dfrac{7}{3}\\
	y=&5x-\dfrac{13}{3}\\
	\end{align*}
	Cioè l'equazione della retta parallela cercata.
	\begin{center}
		\includestandalone[width=.6\textwidth]{terzo/grafici/retta_dis_20}
		%\captionof{figure}{Grafico}\label{fig:EsRiedistanza13}
	\end{center}
	
\end{exercise}
\tcbstoprecording
\newpage
\section{Soluzioni esercizi di riepilogo}
\tcbinputrecords							