\section{Retta intersezioni}
\begin{exercise}
	Dati i punti  $A\coord{-13}{-12}$, $B\coord{-15}{-8}$,trovare l'equazione della retta che passa per questi punti. Trovare le coordinate dei punti di intersezione con gli assi della retta trovata. Disegnare la retta.
	\tcblower
	Dati i punti  $A\coord{-13}{-12}$, $B\coord{-15}{-8}$,trovare l'equazione della retta che passa per questi punti. Trovare le coordinate dei punti di intersezione con gli assi della retta trovata. Disegnare la retta.
	
	Consideriamo l'equazione  $\dfrac{y-y_1}{y_2-y_1}=\dfrac{x-x_1}{x_2-x_1}$
	\begin{align*}
		\intertext{Passaggio per  $A\coord{-13}{-12}$, $B\coord{-15}{-8}$}
		\dfrac{y+12}{-8+12}=&\dfrac{x+13}{-15+13}\\
		\dfrac{y+12}{4}=&\dfrac{x+13}{-2}\\
		-2y-24=&4x+52\\
		4x+2y+76=&0\\
		2x+y+38=&0\\
	\end{align*}
	Otteniamo la retta cercata.
	Intersezione con l'asse $x$ di $2x+y+38=0$
	\begin{align*}
		2x+38=&0\\
		2x=&-38\\
		x=&-19
	\end{align*}
	$A\coord{-19}{0}$
	
	Intersezione con l'asse $y$
	$B\coord{0}{-38}$
	\begin{center}
		\includestandalone[width=.6\textwidth]{terzo/grafici/retta_dis_17}
		%\captionof{figure}{Grafico}\label{fig:EsRiedistanza13}
	\end{center}
\end{exercise}
\begin{exercise}
	Date le rette $y=3x+3$ e $y=-2x-2$ trovare, se esiste, il loro punto di intersezione $A$.
	\tcblower
	Date le rette $y=3x+3$ e $y=-2x-2$ trovare, se esiste, il loro punto di intersezione $A$.
	\begin{align*}
		&\begin{cases}
			y=3x+3\\
			y=-2x-2
		\end{cases}
		&&\begin{cases}
			y=3x+3\\
			3x+5=-2x-2
		\end{cases}\\
		&\begin{cases}
			y=3x+3\\
			5x=-3-2
		\end{cases}
		&&\begin{cases}
			y=3x+3\\
			5x=-5
		\end{cases}\\
		&\begin{cases}
			y=3(-1)+3\\
			x=-1
		\end{cases}
		&&\begin{cases}
			y=0\\
			x=-1
		\end{cases}
	\end{align*}
	$A\coord{-1}{0}$
	\begin{center}
		\includestandalone[width=.6\textwidth]{terzo/grafici/retta_dis_18}
		%\captionof{figure}{Grafico}\label{fig:EsRiedistanza13}
	\end{center}
\end{exercise}
\begin{exercise}
	Date le rette $y=-3x+1$ e $x+y-3=0$ trovare,se esiste, il loro punto di intersezione $A$. 
	\tcblower
	Date le rette $y=-3x+1$ e $x+y-3=0$ trovare,se esiste, il loro punto di intersezione $A$.
	\begin{align*}
		&\begin{cases}
			x+y-3=0\\
			y=-3x+1
		\end{cases}
		&&\begin{cases}
			x-3x+1-3=0\\
			y=-3x+1
		\end{cases}\\
		&\begin{cases}
			-2x-2=0\\
			y=-3x+1
		\end{cases}
		&&\begin{cases}
			y=-1\\
			y=-3(-1)+1
		\end{cases}\\
		&\begin{cases}
			y=4\\
			x=-1
		\end{cases}
	\end{align*}
	$A\coord{-1}{4}$
	\begin{center}
		\includestandalone[width=.6\textwidth]{terzo/grafici/retta_dis_19}
		%\captionof{figure}{Grafico}\label{fig:EsRiedistanza13}
	\end{center}
\end{exercise}
\begin{exercise}
	Date le rette $y=-2x+5$ e $y=x+1$ trovare,se esiste, il loro punto di intersezione $A$. Trovare l'equazione della parallela a $y=5x+4$ che passa per il punto $A$.
	\tcblower
	Date le rette $y=-2x+5$ e $y=x+1$ trovare,se esiste, il loro punto di intersezione $A$. Trovare l'equazione della parallela a $y=5x+4$ che passa per il punto $A$.
	\begin{align*}
		&\begin{cases}
			y=-2x+5\\
			y=x+1
		\end{cases}
		&&\begin{cases}
			-2x+5=x+1\\
			y=x+1
		\end{cases}\\
		&\begin{cases}
			-3x=-4\\
			y=x+1
		\end{cases}
		&&\begin{cases}
			y=\dfrac{4}{3}\\[.5em]
			y=\dfrac{4}{3}+1
		\end{cases}\\
		&\begin{cases}
			y=\dfrac{4}{3}\\[.5em]
			y=\dfrac{7}{3}
		\end{cases}
	\end{align*}
	$A\coord{\dfrac{4}{3}}{\dfrac{7}{3}}$
	Due rette sono parallele se hanno lo stesso coefficiente angolare quindi \[m_1=m_2 \]
	quindi $m_2=5$
	
	Utilizzando l'equazione del fascio di rette, il valore di $m_2$ trovato e le coordinate di $A$ Otteniamo:
	\begin{align*}
		y-\dfrac{7}{3}=&5(x-\dfrac{4}{3})\\
		y=&5x-\dfrac{20}{3}+\dfrac{7}{3}\\
		y=&5x-\dfrac{13}{3}\\
	\end{align*}
	Cioè l'equazione della retta parallela cercata.
	\begin{center}
		\includestandalone[width=.6\textwidth]{terzo/grafici/retta_dis_20}
		%\captionof{figure}{Grafico}\label{fig:EsRiedistanza13}
	\end{center}
	
\end{exercise}