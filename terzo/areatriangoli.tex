\section{Area Triangoli}
\begin{exercise}
	Calcola l'area del triangolo che ha per vertici $A\coord{1}{1}$, $B\coord{6}{1}$ e $C\coord{4}{3}$
	\tcblower
	Calcola l'area del triangolo che ha per vertici $A\coord{1}{1}$, $B\coord{6}{1}$ 
	e $C\coord{4}{3}$
	L'area del triangolo è data da \[A=\dfrac{b\cdot h}{2}\]
	Quindi devo conoscere la lunghezza della base $AB$ e dell'altezza $CH$.
	La base $AB$ è un segmento orizzontale quindi \[b=d(AB)=\left| x_A-x_B\right|=\left| 1-6\right|=\left|-5\right|=5   \]
	Per trovare l'altezza prima troviamo le coordinate del punto $H$. Dato che è sul segmento $AB$ il punto $H$ ha la stessa ordinata di $A$. Inoltre dato che è sul segmento perpendicolare $CH$, $H$ ha la stessa ascissa di $C$. Quindi,  $H\coord{4}{1}$. Abbiamo \[h=d(CH)=\left| y_C-y_H\right|=\left| 1-3\right|=\left|-2\right|=2   \]In conclusione: \[A=\dfrac{b\cdot h}{2}=\dfrac{5\cdot 2}{2}=5\]
	\begin{center}
		\includestandalone[width=.5\textwidth]{terzo/grafici/AreaTriangolo1}
		\captionof{figure}{Aera triangolo}\label{fig:AreaTriangolo1}
	\end{center}
\end{exercise}
\begin{exercise}
	Calcola l'area del triangolo che ha per vertici $A\coord{-3}{1}$, $B\coord{1}{1}$ e $C\coord{3}{4}$
	\tcblower
	Calcola l'area del triangolo che ha per vertici $A\coord{-3}{1}$, $B\coord{1}{1}$ e $C\coord{3}{4}$
	L'area del triangolo è data da \[A=\dfrac{b\cdot h}{2}\]
	Quindi devo conoscere la lunghezza della base $AB$ e dell'altezza $CH$.
	La base $AB$ è un segmento orizzontale quindi \[b=d(AB)=\left| x_A-x_B\right|=\left|-3-1\right|=\left|-4\right|=4   \]
	Per trovare l'altezza prima troviamo le coordinate del punto $H$. Dato che è esterno al segmento $AB$ il punto $H$ si trova sul prolungamento di $AB$ e ha la stessa ordinata di $A$. Inoltre dato che è sul segmento perpendicolare $CH$, $H$ ha la stessa ascissa di $C$. Quindi,  $H\coord{3}{1}$. Abbiamo \[h=d(CH)=\left| y_C-y_H\right|=\left| 4-1\right|=\left|3\right|=3   \]In conclusione: \[A=\dfrac{b\cdot h}{2}=\dfrac{4\cdot 3}{2}=6\]
	\begin{center}
		\includestandalone[width=.5\textwidth]{terzo/grafici/AreaTriangolo2}
		\captionof{figure}{Aera triangolo}\label{fig:AreaTriangolo2}
	\end{center}
\end{exercise}
\begin{exercise}
	Calcola l'area del triangolo che ha per vertici $A\coord{3}{0}$, $B\coord{3}{5}$ e $C\coord{0}{3}$
	\tcblower
	Calcola l'area del triangolo che ha per vertici $A\coord{3}{0}$, $B\coord{3}{5}$ e $C\coord{0}{3}$
	L'area del triangolo è data da \[A=\dfrac{b\cdot h}{2}\]
	Quindi devo conoscere la lunghezza della base $AB$ e dell'altezza $CH$.
	La base $AB$ è un segmento verticale quindi \[b=d(AB)=\left| y_A-y_B\right|=\left| 0-5\right|=\left|-5\right|=5   \]
	Per trovare l'altezza prima troviamo le coordinate del punto $H$. Dato che è sul segmento $AB$ il punto $H$ ha la stessa ascissa di $A$. Inoltre dato che è sul segmento orizzontale $CH$, $H$ ha la stessa ordinata di $C$. Quindi,  $H\coord{3}{3}$. Abbiamo \[h=d(CH)=\left| x_C-x_H\right|=\left| 0-3\right|=\left|-3\right|=3   \]In conclusione: \[A=\dfrac{b\cdot h}{2}=\dfrac{5\cdot 3}{2}=\dfrac{15}{2}\]
	\begin{center}
		\includestandalone[width=.5\textwidth]{terzo/grafici/AreaTriangolo3}
		\captionof{figure}{Aera triangolo}\label{fig:AreaTriangolo3}
	\end{center}
\end{exercise}
\begin{exercise}
	Calcola l'area del triangolo che ha per vertici $A\coord{3}{0}$, $B\coord{3}{3}$ e $C\coord{0}{6}$
	\tcblower
		Calcola l'area del triangolo che ha per vertici $A\coord{3}{0}$, $B\coord{3}{3}$ e $C\coord{0}{6}$
	L'area del triangolo è data da \[A=\dfrac{b\cdot h}{2}\]
	Quindi devo conoscere la lunghezza della base $AB$ e dell'altezza $CH$.
	La base $AB$ è un segmento orizzontale quindi \[b=d(AB)=\left| x_A-x_B\right|=\left|-3-1\right|=\left|-4\right|=4   \]
	Per trovare l'altezza prima troviamo le coordinate del punto $H$. Dato che è esterno al segmento $AB$ il punto $H$ si trova sul prolungamento di $AB$ e ha la stessa ordinata di $A$. Inoltre dato che è sul segmento perpendicolare $CH$, $H$ ha la stessa ascissa di $C$. Quindi,  $H\coord{3}{1}$. Abbiamo \[h=d(CH)=\left| y_C-y_H\right|=\left| 4-1\right|=\left|3\right|=3   \]In conclusione: \[A=\dfrac{b\cdot h}{2}=\dfrac{4\cdot 3}{2}=6\]
	\begin{center}
		\includestandalone[width=.5\textwidth]{terzo/grafici/AreaTriangolo4}
		\captionof{figure}{Aera triangolo}\label{fig:AreaTriangolo4}
	\end{center}
\end{exercise}